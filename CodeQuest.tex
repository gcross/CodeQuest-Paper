%@+leo-ver=4-thin
%@+node:gcross.20090405101642.3:@thin CodeQuest.tex
%@@language latex

%@<< Prelude >>
%@+node:gcross.20090405101642.4:<< Prelude >>
\documentclass[twocolumn,showpacs,preprintnumbers,amsmath,amssymb,nofootinbib,pra,floatfix]{revtex4-1}

\usepackage{mathrsfs,amsthm,clrscode,comment}

\newtheorem{theorem}{Theorem}
\newtheorem{proposition}{Proposition}
\newtheorem{lemma}{Lemma}
\newtheorem{corolary}{Corolary}

\newenvironment{definition}[1][Definition]{\begin{trivlist}
\item[\hskip \labelsep {\bfseries #1}]}{\end{trivlist}}

\newenvironment{example}[1][Example]{\begin{trivlist}
\item[\hskip \labelsep {\bfseries #1}]}{\end{trivlist}}
\newenvironment{remark}[1][Remark]{\begin{trivlist}
\item[\hskip \labelsep {\bfseries #1}]}{\end{trivlist}}

%@-node:gcross.20090405101642.4:<< Prelude >>
%@nl

%@<< Macros >>
%@+node:gcross.20091029172439.1256:<< Macros >>
\newcommand{\lst}{\vec}
\newcommand{\set}{\tilde}

\newcommand{\genfun}{\mathcal{G}}
\newcommand{\pauligroup}{\mathfrak{P}}
\newcommand{\powerset}{\mathcal{P}}
\newcommand{\centralizer}{\mathcal{C}}

%@+leo-ver=4
%@+node:@file /Users/cog/Documents/macros.tex
%@@language latex

\newcommand{\drv}[2]{ \frac{d#1}{d#2} }
\newcommand{\pdrv}[2]{ \frac{\partial #1}{\partial #2} }
\newcommand{\dpdrv}[2]{ \frac{\partial^2 #1}{\partial #2^2} }
\newcommand{\ddrv}[2]{\frac{d^2#1}{d#2^2}}
\newcommand{\ndrv}[3]{\frac{d^{#3}#1}{d#2^{#3}}}
\newcommand{\half}{\frac{1}{2}}
\newcommand{\quarter}{\frac{1}{4}}

\newcommand{\intii}{\int^{\infty}_{-\infty}}

\newcommand{\eqn}[2]{\begin{equation}\label{#1}#2\end{equation}}

\newcommand{\E}[1]{\times 10^{#1}}

\newcommand{\om}{\omega}
\newcommand{\bomega}{\vec{\omega}}
\newcommand{\bom}{\bomega}
\newcommand{\domega}{\dot\omega}
\newcommand{\dom}{\domega}

%\newcommand{\th}{\theta}

\newcommand{\unit}[1]{\,\text{#1}}

\newcommand{\g}{\unit{g}}
\newcommand{\kg}{\unit{kg}}
\newcommand{\m}{\unit{m}}
\newcommand{\um}{\,\mu\!\unit{m}}
\newcommand{\micrometer}{\um}
\newcommand{\mm}{\unit{mm}}
\newcommand{\nm}{\unit{nm}}
\newcommand{\picom}{\unit{pm}}
\newcommand{\km}{\unit{km}}
\newcommand{\cm}{\unit{cm}}
\newcommand{\s}{\unit{s}}
\newcommand{\us}{\,\mu\!\unit{s}}
\newcommand{\ns}{\unit{ns}}
\renewcommand{\min}{\unit{min}}
\newcommand{\N}{\unit{N}}
\newcommand{\Li}{\unit{L}}
\newcommand{\Hz}{\unit{Hz}}
\newcommand{\mHz}{\unit{mHz}}
\newcommand{\kHz}{\unit{kHz}}
\newcommand{\MHz}{\unit{MHz}}
\newcommand{\GHz}{\unit{GHz}}
\newcommand{\J}{\unit{J}}
\newcommand{\kJ}{\unit{kJ}}
\newcommand{\mol}{\unit{mol}}
\newcommand{\K}{\unit{K}}
\newcommand{\W}{\unit{W}}
\newcommand{\kW}{\unit{kW}}
\newcommand{\V}{\unit{V}}
\newcommand{\eV}{\unit{eV}}
\newcommand{\keV}{\unit{keV}}
\newcommand{\MeV}{\unit{MeV}}
\newcommand{\meV}{\unit{meV}}
\newcommand{\ueV}{\unit{ueV}}
\newcommand{\T}{\unit{T}}
\newcommand{\C}{\unit{C}}
\newcommand{\hour}{\unit{hr}}
\newcommand{\dayunit}{\unit{day}}

\newcommand{\tr}{\text{tr}\,}

\newcommand{\vel}{\,\frac{\m}{\s}}

\newcommand{\speedoflight}{3\E{8}\vel}


\renewcommand{\max}{\text{max}}

\renewcommand{\dim}[1]{\left[#1\right]}

\newcommand{\rms}{\text{rms}}

\renewcommand{\deg}{^\circ}

\newcommand{\dx}{\dot x}
\newcommand{\dv}{\dot v}
\newcommand{\ddx}{\ddot x}

\newcommand{\mathboit}{\textbf}

\newcommand{\hH}{\hat H}
\newcommand{\ha}{\hat a}
\newcommand{\hadag}{\hat a^\dagger}

\newcommand{\btheta}{\theta}
\newcommand{\bF}{\mathboit F}
\newcommand{\bL}{\mathboit L}
\newcommand{\br}{\mathboit r}
\newcommand{\bl}{\mathboit l}
\newcommand{\bp}{\mathboit p}
\newcommand{\bc}{\mathboit c}
\newcommand{\bv}{\mathboit v}
\newcommand{\bA}{\mathboit{A}}
\newcommand{\bB}{\mathboit{B}}
\newcommand{\bJ}{\mathboit{J}}
\newcommand{\bD}{\mathboit{D}}
\newcommand{\bS}{\mathboit{S}}
\newcommand{\bT}{\mathboit{T}}
\newcommand{\bI}{\mathboit{I}}
\newcommand{\bR}{\mathboit{R}}
\newcommand{\bK}{\mathboit{K}}
\newcommand{\bH}{\mathboit{H}}
\newcommand{\bM}{\mathboit{M}}
\newcommand{\bG}{\mathboit{G}}
\newcommand{\bP}{\mathboit{P}}
\newcommand{\bE}{\mathboit{E}}
\newcommand{\bV}{\mathboit{V}}
\newcommand{\bQ}{\mathboit{Q}}
\newcommand{\bO}{\mathboit{O}}
\newcommand{\bX}{\mathboit{X}}
\newcommand{\bZ}{\mathboit{Z}}
\newcommand{\bC}{\mathboit{C}}
\newcommand{\bN}{\mathboit{N}}

\newcommand{\bx}{\mathboit x}
\newcommand{\bz}{\mathboit z}
\newcommand{\be}{\mathboit e}
\newcommand{\bg}{\mathboit g}

\newcommand{\bn}{\mathboit n}
\newcommand{\ba}{\mathboit a}
\newcommand{\bb}{\mathboit b}
\newcommand{\bu}{\mathboit u}
\newcommand{\bs}{\mathboit s}

\newcommand{\bi}{\mathboit i}
\newcommand{\bj}{\mathboit j}
\newcommand{\bk}{\mathboit k}

\newcommand{\mO}{\mathscr O}
\newcommand{\mL}{\mathscr L}
\newcommand{\mH}{\mathscr H}

\newcommand{\hx}{\hat{\mathboit x}}
\newcommand{\hy}{\hat{\mathboit y}}
\newcommand{\hz}{\hat{\mathboit z}}
\newcommand{\hr}{\hat{\mathboit r}}
\newcommand{\he}{\hat{\mathboit e}}
\newcommand{\htheta}{\hat{\theta}}

\newcommand{\hse}{\,\hat{\mathboit e}\,}

\newcommand{\va}{\vec{a}}
\newcommand{\vb}{\vec{b}}
\newcommand{\vc}{\vec{c}}
\newcommand{\vd}{\vec{d}}
%\newcommand{\vr}{\vec{r}}
\newcommand{\vx}{\vec{x}}
\newcommand{\vy}{\vec{y}}
\newcommand{\vz}{\vec{z}}
%\newcommand{\vr}{\vec{r}}
\newcommand{\vp}{\vec{p}}
\newcommand{\vq}{\vec{q}}
\newcommand{\vk}{\vec{k}}
\newcommand{\vv}{\vec{v}}
\newcommand{\vu}{\vec{u}}
\newcommand{\vf}{\vec{f}}
\newcommand{\vR}{\vec{R}}
\newcommand{\vP}{\vec{P}}
\newcommand{\vS}{\vec{S}}
\newcommand{\vs}{\vec{s}}
\newcommand{\vL}{\vec{L}}
\newcommand{\vl}{\vec{l}}
\newcommand{\vJ}{\vec{J}}
\newcommand{\vj}{\vec{j}}
\newcommand{\vF}{\vec{F}}
\newcommand{\vE}{\vec{E}}
\newcommand{\vB}{\vec{B}}
\newcommand{\vM}{\vec{M}}
\newcommand{\vA}{\vec{A}}
\newcommand{\vH}{\vec{H}}
\newcommand{\vD}{\vec{D}}
\newcommand{\vsigma}{\vec{\sigma}}
\newcommand{\valpha}{\vec{\alpha}}

\newcommand{\vepsilon}{\vec{\epsilon}}

\newcommand{\dtbx}{\dot \bx}
\newcommand{\ddtbx}{\ddot \bx}
\newcommand{\dtbr}{\dot \br}
\newcommand{\ddtbr}{\ddot \br}
\newcommand{\dtbR}{\dot \bR}
\newcommand{\ddtbR}{\ddot \bR}

\newcommand{\dtx}{\dot x}
\newcommand{\dty}{\dot y}
\newcommand{\dtz}{\dot z}
\newcommand{\dts}{\dot s}
\newcommand{\dtheta}{\dot \theta}
\newcommand{\dttheta}{\dot \theta}
\newcommand{\ddtx}{\ddot x}
\newcommand{\ddty}{\ddot y}
\newcommand{\ddtz}{\ddot z}
\newcommand{\ddts}{\ddot s}
\newcommand{\ddttheta}{\ddot \theta}
\newcommand{\dtr}{\dot r}
\newcommand{\ddtr}{\ddot r}
\newcommand{\dtl}{\dot l}
\newcommand{\ddtl}{\ddot l}

\newcommand{\atan}{\text{atan}\,}
\newcommand{\atanh}{\text{atanh}\,}
\newcommand{\acot}{\text{acot}\,}
\newcommand{\acoth}{\text{acoth}\,}

%\newcommand{\br}{\mathboit r}
\renewcommand{\r}{\br}
%\newcommand{\bv}{\mathboit v}
\renewcommand{\v}{\bv}

\newcommand{\curl}{\nabla\times}

\newcommand{\ip}[2]{\left<#1,#2\right>}
\newcommand{\cip}[2]{\left<#1|#2\right>}
\newcommand{\coip}[3]{\left<#1\left|#2\right|#3\right>}

\newcommand{\ket}[1]{\left|#1\right>}
\newcommand{\bra}[1]{\left<#1\right|}
\newcommand{\ketbra}[2]{\left|#1\right>\left<#2\right|}

\newcommand{\valuepermittivity}{\left(8.85419\E{-12}\frac{\C^2}{\J\m}\right)}
\newcommand{\valuehbar}{\left(1.05457\E{-34}\J\s\right)}
\newcommand{\valueplanck}{\left(6.62607\E{-34}\J\s\right)}
\newcommand{\valuefundamentalcharge}{\left(1.60218\E{-19}\C\right)}
\newcommand{\valuemassofelectron}{\left(9.10939\E{-31}\kg\right)}
\newcommand{\valuemassofproton}{\left(1.67262\E{-27}\kg\right)}
\newcommand{\valuespeedoflight}{\left(3\E{8}\vel\right)}
\newcommand{\valueboltzmann}{\left(1.3807\E{-23}\frac{\J}{\K}\right)}
\newcommand{\valueraleigh}{\left(1.097\E{7}\m^{-1}\right)}
\newcommand{\valuegasconstant}{8.31\frac{\J}{\mol\K}}

\newcommand{\operatormomentum}{\left(\frac{\hbar}{i}\drv{}{x}\right)}

\renewcommand{\exp}[1]{\left<#1\right>}

\newcommand{\sech}{\,\text{sech}\,}

\renewcommand{\choose}[2]{\left(\begin{matrix}#1\\#2\end{matrix}\right)}

\newcommand\cancel{\bgroup \markoverwith{---}\ULon}


\newcommand{\mmat}[4]{
\begin{matrix}
#1 & #2\\
#3 & #4\\
\end{matrix}
}

\newcommand{\bmat}[4]{
\begin{bmatrix}
\,\,\,\, #1 \,\, & \,\, #2 \,\,\,\, \\
\,\,\,\, #3 \,\, & \,\, #4 \,\,\,\, \\
\end{bmatrix}
}


\newcommand{\pmat}[4]{
\begin{pmatrix}
\,\,\,\, #1 \,\, & \,\, #2 \,\,\,\, \\
\,\,\,\, #3 \,\, & \,\, #4 \,\,\,\, \\
\end{pmatrix}
}

\newcommand{\bmattt}[9]{
\begin{bmatrix}
\,\,\,\, #1 \,\, & \,\, #2 \,\, & \,\, #3 \,\,\,\, \\
\,\,\,\, #4 \,\, & \,\, #5 \,\, & \,\, #6 \,\,\,\, \\
\,\,\,\, #7 \,\, & \,\, #8 \,\, & \,\, #9 \,\,\,\, \\
\end{bmatrix}
}


\newcommand{\pmattt}[9]{
\begin{pmatrix}
\,\,\,\, #1 \,\, & \,\, #2 \,\, & \,\, #3 \,\,\,\, \\
\,\,\,\, #4 \,\, & \,\, #5 \,\, & \,\, #6 \,\,\,\, \\
\,\,\,\, #7 \,\, & \,\, #8 \,\, & \,\, #9 \,\,\,\, \\
\end{pmatrix}
}

\newcommand{\bvec}[2]{
\begin{bmatrix}
#1 \\
#2 \\
\end{bmatrix}
}
\newcommand{\bveccc}[3]{
\begin{bmatrix}
#1 \\
#2 \\
#3 \\
\end{bmatrix}
}

\newcommand{\bvecccc}[4]{
\begin{bmatrix}
#1 \\
#2 \\
#3 \\
#4 \\
\end{bmatrix}
}

\newcommand{\pvec}[2]{
\begin{pmatrix}
#1 \\
#2 \\
\end{pmatrix}
}
\newcommand{\pveccc}[3]{
\begin{pmatrix}
#1 \\
#2 \\
#3 \\
\end{pmatrix}
}
\newcommand{\pvecccc}[4]{
\begin{pmatrix}
#1 \\
#2 \\
#3 \\
#4 \\
\end{pmatrix}
}


\newcommand{\btvec}[2]{
\begin{bmatrix}
\,\,\,\, #1 \,\, & \,\, #2 \,\,\,\, \\
\end{bmatrix}
}

\newcommand{\paren}[1]{\left(#1\right)}

\newcommand{\pr}[1]{#1^\prime}
\newcommand{\dpr}[1]{#1^{\prime\prime}}

\newcommand{\halfpi}{\frac{\pi}{2}}
\newcommand{\twopi}{2\pi}

\newcommand{\vn}{\vec\nabla}
\newcommand{\vnabla}{\vec\nabla}

\newcommand{\cippsiket}[4]{\left<\psi_{#1}^{(#2)}|\psi_{#3}^{(#4)}\right>}
\newcommand{\coippsiket}[5]{\left<\psi_{#1}^{(#2)}|#3|\psi_{#4}^{(#5)}\right>}
\newcommand{\psiketord}[2]{\ket{\psi_{#1}^{(#2)}}}
\newcommand{\psibraord}[2]{\bra{\psi_{#1}^{(#2)}}}

\newcommand{\psiketordt}[3][t]{\ket{\psi_{#2}^{(#3)}(#1)}}
\newcommand{\psibraordt}[3][t]{\bra{\psi_{#2}^{(#3)}(#1)}}

\newcommand{\coippsikett}[6][t]{\left<\psi_{#2}^{(#3)}(#1)|#4|\psi_{#5}^{(#6)}(#1)\right>}

\newcommand{\paslash}{\ensuremath \raisebox{0.025cm}{\slash}\hspace{-0.25cm}\partial\/}

\newcommand{\sll}[1]{\rlap{\hbox{$\mskip 1 mu /$}}#1}      % good slash for lower case
\newcommand{\Sl}[1]{\rlap{\hbox{$\mskip 3 mu /$}}#1}      % " upper
\newcommand{\SL}[1]{\rlap{\hbox{$\mskip 4.5 mu /$}}#1}    % " fat stuff (e.g., M)



%%% Local Variables: 
%%% mode: latex
%%% TeX-master: t
%%% End: 
%@-node:@file /Users/cog/Documents/macros.tex
%@-leo

%@-node:gcross.20091029172439.1256:<< Macros >>
%@nl

\begin{document}

%@+others
%@+node:gcross.20090513124712.1:Title Page
\title{CodeQuest}

\author{Gregory M. Crosswhite}
\affiliation{Department of Physics\\ University of Washington\\ Seattle, 98185}

\author{Dave Bacon}
\affiliation{Department of Computer Science \& Engineering \\ Department of Physics \\ University of Washington \\ Seattle, 98185}

%\pacs{03.67.-a}

\email{gcross@phys.washington.edu, dabacon@cs.washington.edu}


\maketitle

\newpage

\tableofcontents
%@-node:gcross.20090513124712.1:Title Page
%@+node:gcross.20090405101642.5:Introduction
In the field of quantum computing, there is a grand battle between the
forces of humankind, which seek to reliably store and manipulate
quantum information, and the forces of nature, which generally seek to
destroy it.  Although armies of experimentalists have made hereoic
efforts to build systems that shield quantum information from harm,
nature inevitably manages to get past these defences from time to time
and strike a blow.  This might seem to paint a grim outlook for the
possibility of building a quantum computer, but happily it turns out
to be the case that one can generally repair damage to quantum
information as long as one knows the exact form that the damage took,
and furthermore that one can build a `trap' --- that is to say, a
\emph{quantum code} --- that tricks nature into giving this
information up.

Now, the nature of codes is that they decouple the space in which our
computation lives from the space in which the physical information is
stored; that is to say, although we design our quantum circuit to
operate on some space of qubits $\mathscr{C}$, each of these qubits
does \emph{not} directly correspond to a physical qubit, but rather
there is some isomorphism that relates the entire space $\mathscr{C}$
to the space of physical qubits, $\mathscr{P}$.  To distinguish
between these two spaces, we shall call the space of qubits in whose
terms the computation is expressed the \emph{computational space}, and
the space of qubits which have physically been built the
\emph{physical space}.

Of course, merely building an isomorphism between these two spaces is
not enough to allow us to correct errors.  For one thing, we need to
add extra qubits to the computational space that contain a record of
the damage that we can read out; thus, we shall say that the full
computational space is $\mathscr{C}:=\mathscr{R}\times\mathscr{Q}$,
where the qubits that live in $\mathscr{R}$ have the role of keeping a
record of the errors that have been introduces, and the qubits that
live in $\mathscr{Q}$ are the qubits in whose terms our quantum
algorithm is expressed.  Since we are only performing measurements on
$\mathscr{R}$, we can effectively ignore all operators except, say,
the $Z$ measurement operator for each qubit; this set of commuting
operators allows us to completely measure the state of qubits in
$\mathscr{R}$.  In order to build the `trap' element into our system,
we need to ensure that whenever nature strikes at the physical space
$\mathscr{P}$, it is isomorphic to a strike on the computational space
that leaves a \emph{measureable} record in $\mathscr{R}$, which means
in particular that it is isomorphic to an operator that must
\emph{anti-commute} with the $Z$ operator (or whatever else we have
chosen to be our basis of measurement) of one of the qubits in
$\mathscr{R}$.  Note that although we speak of measuring the qubits in
$\mathscr{R}$, they of course cannot be measured directly, but instead
we take the measurement operator of interest in $\mathscr{R}$ and
measure the \emph{isomorphic} operator in the physical space
$\mathscr{P}$; this isomorphic operator is referred to as a
\emph{stabilizer}, and the full set of operators on $\mathscr{P}$
which are isomorphic to our chosen measurement operators on
$\mathscr{R}$ are referred to as the \emph{stabilizers} of the code.

Up to this point, the formalism we have described is known as
\emph{stabilizer codes} and its essential characteristic is the
forcing of every qubits in $\mathscr{R}$ to always have a definite
value in some basis by performing continuous measurement.  What if,
however, we relaxed this constraint and only continuously measured
some of the qubits in $\mathscr{R}$?  That is to say, what if we split
the qubits in $\mathscr{R}$ into two catagories: \emph{stabilizer
qubits} whose states we care about and force to always have a definite
value in some basis through continuous measurement, \emph{gauge
qubits} whose states we do not care about.  (The latter get their name
from the fact that they provide a `gauge' degree of freedom, i.e. a
degree of freedom that is irrelevent to us.)  Then we would have that
$\mathscr{R}=\mathscr{S}\times \mathscr{G}$, where $\mathscr{S}$ is
the space in which the stabilizer qubits live, and $\mathscr{G}$ is
the space in which the so-called gauge qubits live; such a scheme is
known as a \emph{subsystem code}.  In this case, we shall use the term
\emph{stabilizers} to denote the set of operators in $\mathscr{P}$
which are isomorphic to our chosen measurement operators of interest
in $\mathscr{S}$.

At first there might not seem to be an advantage to this approach,
since it essentially means adding qubits to our code that are
`wasted'; however, in practice this can actually make our code easier
to implement in a physical system.  The reason for this is that often
the measurement of stabilizers requires performing operations that
involve several qubits at once, which can be difficult or impossible
to implement\footnote{See, for example, the \emph{toric code} [ref],
which uses 4-qubit measurements.}  However, there are ways that by
adding additional qubits, one can instead use a set of, say, 2-qubit
operators whose simultaneous measurement results in an effective
measurement of all of the stabilizers so that the stabilizer qubits
are all collapsed to definite values in our chosen basis\footnote{For
examples of this, see the compass model code [ref].}.

What makes this approach powerful is that we no longer need our
measurements on the physical system to commute with each other, as
long as they all commute with the stabilizers, since then the fact
that they do not commute only affects the gauge qubits, which we do
not care about.  In fact, it is so powerful that any set of
measurements that are members of the Pauli group on the physical
qubits can be used to implement a subsystem code, and in fact we can
compute the code that it implements, as we shall prove in this paper;
of course, the resulting code might not be useful --- since among
other possibilities, it might be that it has no room for encoding the
quantum information that we want to store --- but it definitely
exists.  This fact invites an approach to finding useful subsystem
codes that is in many ways opposite to the approach commonly taken:
rather than coming up with codes and then trying to figure out how
they might be physically implemented, why not start with a class of
physical implementations and search within it for useful subsystem
codes?  This is the approach that we explore in this paper.

In the first section, we shall formally prove that every set of
measurements that are members of the Pauli group acting on the system
give rise to a subsystem code, and we shall in the process present
algorithms for computing this code (or at least, for computing one
such code, since it is not unique) and its distance.  In the second
section, we shall present numerical results obtained by applying a
code implementing this algorithm to explore systems built using
lattices that take the form of te 11 regular tilings.  In the third
section, we shall present an algorithm for seaching over all of the
systems that can be implemented by using arbtitrary Ising interactons
with the structure of a graph, and then we shall present the results
that we have obtained from our searches.
%@-node:gcross.20090405101642.5:Introduction
%@+node:gcross.20090423002455.2:Algorithm
\section{Algorithm}

%@+node:gcross.20090423002455.3:Stabilizers and gauge qubits
\subsection{Construction of the subsystem code}

\begin{remark}
This subsection describes by way of a constructive proof how one can compute the quantum subsystem code implemetable by a the set of measurement operators.  For a listing of pseudo-code that implements the algorithm described in this proof, see Table \ref{code-algorithm} near the end of this subsection.
\end{remark}
%@+node:gcross.20090520163423.14:Introduce main theorem
Although conceptually a subsystem code is an isomorphism $T$ such that  $\mathscr{P}\approx^T \mathscr{S}\times\mathscr{G}\times\mathscr{Q}$ --- that is, an isomorphism between the \emph{physical} space of qubits and the \emph{computational} space of qubits in whose terms our computation is actually expressed, we do not need to actually construct this isomorphism in order to be able to use the code.  Since all of our work will be done on the physical system anyway, it suffices to know the operators in the physical space $\mathscr{P}$ that are isomorphic to the qubit measurement operators of interest in the computational space $\mathscr{S}\times\mathscr{G}\times\mathscr{Q}$, and it is exactly the operators on $\mathscr{P}$ that the algorithm we present shall compute\footnote{If one really wanted to, one could explicitly construct the isomorphism $\mathscr{T}$ from these operators by computing the unitary operator which simultaneously diagonalizes a the maximal subset of commuting measurements from this set of operators on $\mathscr{P}$, but in practice this is not particularly useful.}.

When one wants to define a qubit in terms of its measurement operators, it suffices to define two operators that anti-commute with each other but which commute with all of the others measurement operators that have been defined, since this gives us the $X$ and $Z$ measurements on the qubit which are sufficient to generate the full $Pauli$ group (minus phases).  Since working with such pairs of operators shall be a common theme in our algorithm, we shall introduce the following definition in order to simplify the language used to describe them.

\begin{definition} A pair of operators is a \emph{conjugal pair in relation to the set} $\set X$ when each of the operators in the pair commutes with every operator in $\set X$ except for its \emph{conjugal partner} --- that is, the other operator in the conjugal pair --- should its conjugal partner be a member of $\set X$.
\label{conjugal-pair-definition}
\end{definition}

Note that we have explicitly not required that the operators in the conjugal pair be members of $\set X$ in order to be a conjugal pair in relation to it.  However, should both operators be members of $\set X$, then neither operator can belong to a different conjugal pair with respect to $\set X$, since in that case there would be an operator in $\set X$ (namely, its original conjugal partner) with which it anti-commutes that was not its conjugal partner in the new pair, leading to a contradiction.

For convenience, we introduce the following additional definitions:

\begin{definition}

\begin{enumerate}
\item $\pauligroup$ is the group of Pauli operators --- that is, the group of tensor products of the (unnormalized) Pauli matrices --- acting on the physical space $\mathscr{P}$;
\item $\powerset(\set S)$ is the power set of $\set S$, i.e. the set of all subsets of $\set S$; and
\item $\centralizer_\mathfrak{G}(\set S)$ is the centralizer of $\set S$, that is the subgroup of elements in $\mathfrak{G}$ which commute with $\set S$;
\item the function $\genfun:\powerset(\pauligroup)\to\powerset(\pauligroup)$ is defined such that $\genfun(\set S)$ is the set of all possible products of operators in $\set S$ --- that is, it is the set \emph{generated} by $\set S$.
\end{enumerate}

\end{definition}

We now introduce the main theorem of this subsection.

\begin{theorem} \label{theorem-SG} Suppose we are given a sequence of Pauli operators, $\lst O$.  Then there exist sets of Pauli operators $\set S\subseteq\pauligroup$, $\set G\subseteq\pauligroup$, and $\set L\subseteq\pauligroup$ such that
\begin{enumerate}
\item each of the operators in $\set S \cup \set G \cup \set L$ is independent from the rest --- i.e., no operator in this (unioned) set can be written as a product of other operators in the set;
\item each operator in $\set L \cup \set G$ is a member of a conjugal pair in relation to $\set S \cup \set G \cup \set L$;
\item $\genfun(\set S \cup \set G)=\genfun\paren{\{\lst O_i\}}$\footnote{Here we use the notation $\{\vec{O}_i\}$ to refer to the set of elements in the sequence $\vec{O}$.};
\item and $\genfun(\set S \cup \set G \cup \set L)=\centralizer_\pauligroup(\set S)$
\end{enumerate}
\end{theorem}

\begin{remark}
The main work in the proof of this theorem will be performed by proving several related propositions.  First we shall show how the set $\set G$ and a sequence $\lst S$ are constructed from the sequence of operators $\lst O$.  Since we want our stabilizers to form an independent set of operators, we shall then show that through a Gaussian elimination procedure it is possible to extract a list of independent operators from a sequence $\lst S$ resulting in a set $\set S$.  Finally, we shall show how using this same Gaussian elimination procedure we can transform a subset of the operators of $\set S\cup\set G$ into a form that makes it trivial to compute the logical qubit operators $\set L$.
\end{remark}
%@nonl
%@-node:gcross.20090520163423.14:Introduce main theorem
%@+node:gcross.20090519160701.3:Construction of sequences
\begin{proposition} \label{proposition-SG} Suppose that we are given a sequence of Pauli operators $\lst O\subseteq \pauligroup$.  Then there exists a sequence of Pauli operators $\lst S\subseteq\pauligroup$ and a set of Pauli operators $\set G\subseteq\pauligroup$ such that
\begin{enumerate}
\item all of the operators in $\lst S$ commute with each other and also all of the operators in $\lst G$; \label{stabs-commute-with-G}
\item each operator in $\set G$ is a member of a \emph{conjugal pair} (Definition \ref{conjugal-pair-definition}) in relation to $\{\lst S_i\} \cup \set G $ \label{conjugal-pairs-commute-with-SAG}; and
\item $\genfun\paren{\{\lst S_i\}\cup \set G}=\genfun\paren{\{\lst O_i\}}$ \label{SAG-spans-all}.
\end{enumerate}
\end{proposition}

\begin{proof}
Proof by induction.  For the base case, note that if $\lst O$ is empty then $\lst S:=\emptyset$ and $\set G:=\emptyset$ trivially satisfy all properties.

Now assume that the proposition holds for a sequence of length $n-1$, and consider a sequence of operators $\lst O$ of length $n$.  By the inductive hypothesis, we know that there is a sequence $\lst S'$ and a set $\set G'$ satisfying the properties above for the subsequence of $\lst O$ consisting of the first $n-1$ operators.  Let $o:=\lst O_n\cdot \prod_{g\in \set G, \{\lst O_n,g\}=0} \text{conj}_{\set G}(g)$ --- that is, the product of $\lst O_n$ with the conjugal partner of every operator in $\set G$ with which $\lst O_n$ anti-commutes.  This definition guarantees that $o$ commmutes with every operator in $\set G$;  furthermore, we can obtain $\lst O_n$ back from $o$ since every operator in $\set G$ squares to the identity and thus $\lst O_n=o\cdot \prod_{g\in \set G, \{\lst O_n,g\}=0} \text{conj}_{\set G}(o)$; therefore we conclude that $\genfun\paren{\{\lst S'_i\} \cup \set G' \cup \{o\}}=\genfun\paren{\{\lst O_i\}}$.

If $o$ commutes with every operator in $\lst S'$, then set
$$\lst S_i :=
\begin{cases}
\lst S'_i & i \le n-1 \\
o & i = n
\end{cases}
$$
and $\set G := \set G'$, and we are done.  Otherwise, let $s$ be some operator in $\lst S'$ that anti-commutes with $o$, $\set G:=\set G'\cup \{s,o\}$
\footnote{Observe that neither $o$ nor $s$ can be present in $\set G'$ since they commute with every operator in $\set G'$, so the new set $\set G:=\set G'\cup \{s,o\}$ gives us a strictly larger set.  This fact is irrelevant far as the proof is concerned, but it has the important consequence that a computer code implementing the algorithm described by this proof can append $s$ and $o$ to a list of gauge operators and assume that this list continues to form a set (i.e., a sequence without duplicates) without having to explicitly check for this.}, $\lst S_i'' := f(\lst S'_i)$, and $\lst S$ be the subsequence of $\lst S''$ with the identity operators removed, where
$$
f(s') :=
\begin{cases}
s'\cdot s & \{s',o\}=0\\
s' & \text{otherwise}
\end{cases}.
$$
Observe that by this definition, all of the operators in $\lst S$ commute with every operator in $\set G$, so property \ref{stabs-commute-with-G} is satisfied.  Since the only difference between $\set G'$ and $\set G$ is the addition of $s$ and $o$, which form a conjugal pair with respect to $\{\lst S_i\} \cup \set G$, we conclude that property \ref{conjugal-pairs-commute-with-SAG} is satisfied.
Lastly, since $s\in \set G$, we can form any operator in $\lst S'$ with products of operators in $\lst S$ and $\set G$, so therefore $\genfun\paren{\{\lst S_i\} \cup \set G}=\genfun\paren{\{\lst S'_i\} \cup G' \cup \{s,o\}}=\genfun\paren{\{\lst O_i\}}$, and so the final property is satisfied.

We conclude by noting that since all of the operators in $\lst S$ and $\set G$ were formed from products of operators in $\lst O$, which are Pauli operators (i.e., members of the group $\pauligroup$), they are Pauli operators themselves.
\end{proof}
%@-node:gcross.20090519160701.3:Construction of sequences
%@+node:gcross.20090519160701.4:Making them independent
\begin{remark}
A consequence of not requiring independence of the operators in $\lst O$ is that the operators $\lst S$ given by Proposition \ref{proposition-SG} are not necessarily independent.  Happily, since all of these operators can be expressed as tensor products of Pauli operators, we can construct a set of independent operators by performing an analog of Gaussian elimination.
\end{remark}

\begin{proposition}
\label{make-independent-using-elimination}
Suppose that we have been given a sequence of Pauli operators which commute with each other, $\lst R$.  Then there exists
\begin{enumerate}
\item a sequence $\lst S$ of $n$ independent operators such that $\genfun\paren{\{\lst S_i\}}=\genfun\paren{\{\lst R_i\}}$,
\item a sequence of $n$ integers without duplicates in the inclusive range $1\dots n$,
\item and a map $p:\{1\dots n\} \to \{0,1\}$ such that $\lst S_i$ is the only operator in $\lst S$ that anti-commutes with $P_{k_i}^{[p(i)]}$, where $P_k^{[0]}:=X_k$ and $P_k^{[1]}:=Z_k$.
\end{enumerate}
\end{proposition}

\begin{proof}
Proof by induction.  For the base case, we observe that if $\lst R$ is empty, then the trivial sequences $\lst S:=\emptyset$ and $\lst k :=\emptyset$ and the trivial function $p:\emptyset\to\emptyset$ satisfy the requirements.

Now suppose that we know the proposition holds for sequences of length $N-1$, and we are given a sequence $\lst S$ of length $N$.  By our inductive hypothesis, we can apply the proposition to the first $N-1$ operators in $\lst R$ obtain sequences $\lst S'$ and $\lst k'$ of length $n-1$\footnote{$n\ne N$ in general}, and a map $p':\{1\dots n-1\}\to \{0,1\}$ which all satisfy the respective properties of the theorem.  Let $$s:=\lst R_N\cdot \prod_{i=1\dots n-1, \,\,\left\{\lst R_N,P_{k'_i}^{[p(i)]}\right\}=0} \lst S'_i.$$  We know that $s$ commutes with every operator in $\lst S'$ because both $s$ and every operator in $\lst S'$ are equal to products of operators in $\lst R$, which all commute with each other.  Furthermore, since $s$ is a product of $\lst R_N$ and a factor of $\lst S'_i$ for every $i$ such that $\lst R_N$ and $P_{k'_i}^{[p'(i)]}$ anti-commute, and we know that $\lst S_i'$ is the only operator in $\lst S'$ that anti-commutes with $P_{k'_i}^{[p'(i)]}$ for $i=1\dots n-1$, it is therefore the case that $s$ commutes with every member of the set $\{P_{k'_i}^{[p'(i)]}\}_{i=1\dots n-1}$.  Finally, since $s$ is a product of $\lst R_N$ and operators in $\lst S'$, we can obtain $\lst R_N$ entirely from products of operators in $\{\lst S'_i\} \cup \{s\}$, and so $\genfun\paren{\{\lst S'_i\} \cup \{s\}}=\genfun\paren{\{\lst R_i\}}$.

If $s$ is the identity operator, then let $\lst S:=\lst S'$ and $p:=p$ and we are done.  Otherwise, we shall now show that there must exist integers $j\in\{1,\dots,N\}\backslash\{\lst k'_i\}$ and $l\in\{0,1\}$ such that $s$ anti-commutes with $P_{j}^{[l]}$, by demonstrating that if this were not the case then $s$ would have to anti-commute with some element in $\lst S'$, leading to a contradiction.

Assume that $s$ commutes with every operator in the set $\left\{P_j^{[l]}:\quad j\in\{1,\dots,N\}\backslash\{\lst k'_i\}, \quad l\in\{0,1\}\right\}.$  Recalling that $s$ is a member of the Pauli group and thus a tensor product of single-particle Pauli spin matrices, and also that $s$ commutes with every member of the set $\{P_{\lst k'_i}^{[p'(i)]}\}_{i=1\dots n-1}$, we see therefore that $s$ must be a product of elements from this set --- that is, there is some subset $\emptyset \ne \set F \subseteq \{P_{\lst k'_i}^{[p'(i)]}\}_{i=1\dots n-1}$ such that $s=\prod_{o\in \set F} o$.  However, from our inductive hypothesis we know that for every operator $f\in\set F$ there is an operator $s'\in\lst S'$ that anti-commutes with $f$ but commutes with the operators in $\set F\backslash\{f\}$.  Since $s$ is therefore a product of a single operator that anti-commutes with $s'$ and more operators that commute with $s'$, we conclude that $s$ and $s'$ anti-commute, which contradicts our earlier conclusion that $s$ commutes with every operator in $\lst S'$.

Now that we have shown that there exist integers $j\in\{1,\dots,N\}\backslash\{\lst k'_i\}$ and $l\in\{0,1\}$ such that $s$ anti-commutes with $P_{j}^{[l]}$, in terms of these integers we define
$$
\begin{aligned}
\lst S_i &:= 
\begin{cases}
\begin{cases}
\lst S'_i \cdot s & \{\lst S_i',P_j^{[l]}\}=0 \\
\lst S'_i & \text{otherwise}
\end{cases} & 1\le i\le n-1 \\
S' & i=n
\end{cases}, \\
\lst k_i &:=
\begin{cases}
\lst k'_i & 1 \le i \le n-1 \\
j & i=n
\end{cases},\quad \text{and} \\
p(i) &:=
\begin{cases}
p'(i) & 1 \le i \le n-1\\
l & i=n
\end{cases},
\end{aligned}
$$ and we are done.
\end{proof}
%@-node:gcross.20090519160701.4:Making them independent
%@+node:gcross.20090527164539.1:Construct the logical operators
\begin{remark}
Proposition \ref{make-independent-using-elimination} is good for more than computing an independent set of generators from a commuting list of operators;  it is also the key ingrediant in computing the logical qubit operators.
\end{remark}

\begin{proposition}
\label{construction-of-logicals}
Suppose that we have been given the objects described in 1-3 of Proposition \ref{make-independent-using-elimination}.  Let $\set S := \{\vec S_i\}_i.$  Then there exists a set of operators $\set L$ such that
\begin{enumerate}
\item \label{L-are-independent} the operators in $\set S\cup\set L$ are independent;
\item \label{L-are-conjugal-pairs} every operator in $\set L$ is a member of a conjugal pair with respect to $\set S\cup\set L$;
\item \label{L-completes-the-generators} $\genfun\paren{\set S\cup\set L}=\centralizer_\pauligroup(\set S)$ --- that is, the set generated by $\set S\cup\set L$ is equal to the set of Pauli operators that commute with $\set S$.
\end{enumerate}
\end{proposition}

\begin{proof}
Recalling that $n$ is the number of elements in $\lst S$ (and $\set S$), let $\lst l$ be some sequential ordering of $\{1 \dots N\}\backslash\{\vec k_i\}_i$, and then let $\set L:=\{\lst A_i\}_i\cup\{\lst B_i\}_i$ where
$$
\begin{aligned}
\lst A_i &:= P_{\lst l_i}^{[1]}\cdot \prod_{j=1\dots n\atop \{P_{l_i}^{[1]},\lst S_j\}=0} P_{\lst k_j}^{[s(j)]},\\
\lst B_i &:= P_{\lst l_i}^{[0]}\cdot \prod_{j=1\dots n\atop \{P_{l_i}^{[0]},\lst S_j\}=0} P_{\lst k_j}^{[s(j)]}.\\
\end{aligned}
$$

To see that property \ref{L-are-independent} is satisfied, observe the following.  First, the operators in $\set L$ are independent from the operators in $\set S$ since none of them is the identity operator and they all commute with every operator in $\{P_{\lst k_i}^{[s(i)]}\}_{i=1 \dots n}$.  Second, they are independent from each other since for every $i=1 \dots |\lst l|$ we have that $\vec A_i$ is the only operator that anti-commutes with $P_{\lst l_i}^{[0]}$ and $\lst B_i$ is the only operator that anti-commutes with $P_{\lst l_i}^{[1]}$.  Thus we conclude that all of the operators in $\set S\cup\set L$ are independent.

Next, to see that property \ref{L-are-conjugal-pairs} holds, observe that for every choice of operators $\lst A_i$ and $\lst S_j$ we have (by intentional construction) that $\lst S_j$ either anti-commutes with two of the operators in the product forming $\lst A_i$ or none at all, and so $[\lst S_i,\lst A_j]=0$ for all $i=1\dots n$ and $j=1\dots |\lst l|$;  by the same reasoning we see also that $[\lst S_i,\lst B_j]=0$ for all $i=1\dots n$ and $j=1\dots |\lst l|$.  Furthermore, each operator $\lst A_i$ commutes with every operator in $\set L$ except for its conjugal partner $\lst B_i$, since the only factor in $\lst A_i$ that could anti-commute with a factor contained within another operator in $\set L$ is $P_{l_i}^{[1]}$, and $\lst B_i$ is the only operator in $\set L$ that contains a factor $P_{l_i}^{[0]}$ that anti-commutes with $\lst X_{l_i}$;  reversing this argument, we also see that $\lst B_i$ commutes with every operator in $\set L$ except for $\lst A_i$.  Thus, every operator in $\set L$ is a member of a conjugal pair with respect to $\set L\cup\set S$.

Finally, to see that property \ref{L-completes-the-generators} holds, observe that since the operators in $\set S$ commute they can therefore be simultaneously diagonalized, which means that there is an automorphism on $\pauligroup$ that takes $\lst S_i\mapsto P_i^{[1]}$ for every $i=1 \dots n$.  The only operators that commute with every such $P_i^{[1]}$ are those which do not contain any factor of $P_i^{[0]}$ for $i=1 \dots n$, and so $\centralizer_\pauligroup\paren{\{P_i^{[0]}\}_{i=1\dots n}} = \genfun\paren{\{P_i^{[1]}\}_{i=1 \dots n}\cup \{P_i^{[l]}\}_{i=n+1 \dots N, \,\, l=0,1}}$, which has $2N-n$ generators.  Since the automorphism preserves the number of generators in the centralizer, we thus conclude that $\centralizer_\pauligroup(\set S)$ has exactly $2N-n$ generators.  Since $\set S\cup\set L$ contains independent operators which commute with every member of $\set S$, and furthermore $|\set S\cup\set L|=2N-n$, we thus conclude that $\genfun\paren{\set S\cup\set L}=\centralizer_\pauligroup(\set S)$.
\end{proof}
%@nonl
%@-node:gcross.20090527164539.1:Construct the logical operators
%@+node:gcross.20090520163423.15:Prove main theorem
With these building blocks in place, we now prove the main theorem:

\begin{proof}[Theorem \ref{theorem-SG}]
By Proposition \ref{proposition-SG}, we know that there exists a list of operators $\lst S$ and a set of independent operators $\set G$ satisfying the properties that are listed there.  By Proposition \ref{make-independent-using-elimination}, we know that there is an independent set of operators $\set S$ that generate the same subgroup as $\lst S$.  

Now let $\set F$ be a maximal subset of commuting operators in $\set G$ --- i.e., for each conjugal pair in $\set G$ take one of the two operators --- and then let $\set O := \set F \cup \set S$.  Since all of the operators in $\set O$ commute, we apply Proposition \ref{make-independent-using-elimination} again to conclude the existance of the objects listed there, and then we immediately apply Proposition \ref{construction-of-logicals} to show that a set $\set M$ exists with the properties listed there.  We are not done yet, however, since there might be operators in $\set G$ with which operators in $\set M$ anti-commute, so we let
$$\set L := \{m\cdot\prod_{f\in \set F\atop \{M,\text{conj}_{\set G}(f)\}=0} f :\quad m \in\set M\}$$
where $\text{conj}_{\set G}(F)$ is the conjugal partner of $F$ in the set $\set G$.  This guarantees that the operators in $\set L$ commute with every operator in $\set S\cup\set G$, and so we are done.
\end{proof}
%@-node:gcross.20090520163423.15:Prove main theorem
%@+node:gcross.20090511123440.5:Pseudo-code
\begin{remark}
A pseudo-code representation of the algorithm described by Theorem \ref{theorem-SG} is given in Table \ref{SG-algorithm}.
\end{remark}

\begin{table}
\label{SG-algorithm}
\begin{codebox}
\Procname{$\proc{Compute-Subsystem-Code}(\lst O)$}
\li $\lst S \gets []$
\li $\lst G\gets []$
\li \For $o \gets \lst O$ \label{li:csg-main-loop-start}
\li \Do
\li      \For $(g_X, g_Z) \gets \lst G$ %\Comment i.e., iterate over conjugal pairs in $\lst G$
\li      \Do
\li        \kw{if}  $\func{anti}(o,g_X)$ \kw{then} $o \gets o \cdot g_Z$
\li        \kw{if} $\func{anti}(o,g_Z)$ \kw{then} $o \gets o \cdot g_X$
          \End 
\li      \kw{if} \text{$o$ is identity} \kw{then} \Goto \ref{li:csg-main-loop-start}
\li      \For $s \gets \lst S$
\li      \Do
\li        \kw{if} $\func{anti}(o,s)$ \kw{then} \Goto \ref{li:make-into-gauge}
          \End
\li      \Goto \ref{li:csg-main-loop-start}
\li      $\lst G \gets \lst G \cup [(o,s)]$ \label{li:make-into-gauge}
\li      $i \gets 1$ 
\li      \For $s' \gets \lst S$ \label{li:kill-redundant-stabs}
\li      \Do
\li            \kw{if} $s'=s$ \kw{then} \Goto \ref{li:kill-redundant-stabs}
\li            \If $\func{anti}(s',o)$
\li            \Then
\li                  $\lst S[i] \gets s' \cdot s$
\li            \Else
\li                  $\lst S[i] \gets s$
                 \End
\li            $i \gets i+1$
             \End
\li        delete $\lst S[i\dots|\lst S|]$
      \End
\li $\lst I \gets []$
\li $\lst P \gets []$
\li \kw{call} $\proc{Gaussian-Elimination}(\lst S,1,\lst I,\lst P)$
\li $\lst T \gets \lst S \cup [g_X | (g_X,g_Z) \in \lst G]$
\li \kw{call} $\proc{Gaussian-Elimination}(\lst T,|\lst S|+1,\lst I,\lst P)$
\li $\lst L \gets []$
\li \For $i\gets 1\dots\,\,\text{number of physical qubits}$ \label{li:compute-logicals-loop}
\li \Do
\li     \kw{if} $i\in\lst I$ \kw{then} \Goto \ref{li:compute-logicals-loop}
\li     $l_X \gets X_i$
\li     $l_Z \gets Z_i$
\li     \For $(j,p,t)\gets (\lst I,\lst P,\lst T)$
\li     \Do
\li        \If $p=0$
\li        \Then
\li            \kw{if} $\func{anti}(t,X_j)$ \kw{then} $l_X \gets l_X \cdot Z_j$
\li            \kw{if} $\func{anti}(t,Z_j)$ \kw{then} $l_Z \gets l_Z \cdot Z_j$
\li        \Else
\li            \kw{if} $\func{anti}(t,X_j)$ \kw{then} $l_X \gets l_X \cdot X_j$
\li            \kw{if} $\func{anti}(t,Z_j)$ \kw{then} $l_Z \gets l_Z \cdot X_j$
             \End
          \End 
\li     \For $(g_X,g_Z)\in\lst G$
\li     \Do
\li         \kw{if} $\func{anti}(l_X,g_Z)$ \kw{then} $l_X \gets l_X \cdot g_X$
\li         \kw{if} $\func{anti}(l_Z,g_Z)$ \kw{then} $l_Z \gets l_Z \cdot g_X$
          \End
      \End
\li \Return $(\lst S,\lst G,\lst L)$
\end{codebox}
\caption{Algorithm which computes the subsystem code generated by a given list of measurement operators $\lst O$.}
\end{table}
%@-node:gcross.20090511123440.5:Pseudo-code
%@-node:gcross.20090423002455.3:Stabilizers and gauge qubits
%@+node:gcross.20090511123440.3:Optimal generators
\subsection{Optimal generators}

\label{optimal-generators}

%@+others
%@+node:gcross.20091221145013.1274:Lemma:  Recombining generators
In general there are multiple sets of operators that satisfy the properties of \ref{theorem-SG}, as is illustrated by the following Lemma:

\begin{lemma}
\label{combining-pairs}
Given conjugal pairs $Q:=(a,b)$ and $R:=(c,d)$ in relation to some set $\set X$, we have that
\begin{enumerate}
\item the pairs $Q':=(a\cdot c,b)$ and $R':=(c,d\cdot b)$ are conjugal pairs with respect to $\set X \backslash \{Q,R\} \cup \{Q',R'\}$; and
\item $\genfun\paren{\{a,b,c,d\}}=\genfun\paren{\{a\cdot c,b,c,d\cdot b\}}$.
\end{enumerate}
\end{lemma}

\begin{proof}
\begin{enumerate}
\item Since $[a,c]=[a,d]=[b,c]=[b,d]=\{a,b\}=\{c,d\}=0$, we see therefore that $[a\cdot c,c]=[a\cdot c,d\cdot b]=[b,c]=[b,d\cdot b]=\{a\cdot c,b\}=\{c,d\cdot b\}=0$.  Furthermore, since $a$, $b$, $c$ and $d$ commute with every operator in $\set X\backslash \{Q,R\}$, so do $a\cdot c$ and $d\cdot b$.
\item Since $b$ and $c$ are Pauli operators and thus square to the identity, we have that $a\cdot c\cdot c=a$ and $d\cdot b\cdot b=d$, and so $\genfun\paren{\{a,b,c,d\}}=\genfun\paren{\{a\cdot c,b,c,d\cdot b\}}$.
\end{enumerate}
\end{proof}
%@-node:gcross.20091221145013.1274:Lemma:  Recombining generators
%@+node:gcross.20091221145013.1275:Definition: Undetectable error
As a result of this lemma, we see that we can take pairs of arbitrary conjugal pairs from sets $\set G$ and $\set L$ of Theorem \ref{theorem-SG} and replace them with different pairs per the recipe in Lemma \ref{combining-pairs} such that the properties of the theorem still hold.  So given that these sets are not unique, the natural question is:  What is the best choice of $\set G$ and $\set L$?  To answer this, we observe that another criteria we would like for our code to satisfy is that it be as robust to errors as possible;  in particular, we seek to maximize the difficulty of \emph{undetectable errors}, which is defined as follows:

\begin{definition}
Given a set $\set S\subseteq\pauligroup$ and operators $l\in\pauligroup$ and $e\in\centralizer_\pauligroup(\set S)$ which anti-commute (i.e., $\{l,e\}=0$), we say that $e$ is an \emph{undetectable error with respect to} $\set S$ \emph{acting on} $l$.
\end{definition}
%@nonl
%@-node:gcross.20091221145013.1275:Definition: Undetectable error
%@+node:gcross.20091221145013.1276:Definition: Weight
We assume that the `difficulty' of an interaction between our physical system and its environment is related to the number of physical qubits in our system that are participating in the interaction.  Thus, the natural metric for measuring the relative difficulty of an error is given by its weight, which recall is defined as follows:

\begin{definition}
Given an operator $p\in\pauligroup$---which recalls means that $p$ is the tensor product of single-qubit Pauli unnormalized spin matrices---the \emph{weight} of $p$ is the number of single-qubit operators in the product which are non-trivial (i.e., not the identity).  So for example, the weight of $I\otimes I\otimes I$ is 0, the weight of $I\otimes Z\otimes I\otimes X$ is 2, and the weight of $Z\otimes X\otimes Y$ is 3.
\end{definition}
%@nonl
%@-node:gcross.20091221145013.1276:Definition: Weight
%@+node:gcross.20091221145013.1277:Definition: Additional notation
For convenience, we introduce the following additional notation:

\begin{definition}

\begin{itemize}
\item the function $w:\pauligroup\to \mathscr{N}$ is defined such that $w(o)$ gives the weight of $o$;
\item the function $e_{\set S}:\centralizer_\pauligroup\paren{\set S}\to \centralizer_\pauligroup\paren{\set S}$ is defined such that $e_{\set S}(l)$ is the minimizer of $w$ over the set $\left\{o: o\in \centralizer_\pauligroup\paren{\set S}, \{o,l\}=0\right\}$ --- that is, it gives the undetectable error with respect to $\set S$ acting on $l$ that is of minimum weight;
\item the function $\om_{\set S}:\centralizer_\pauligroup\paren{\set S}\to\mathscr{N}$ is defined such that $\om_{\set S}:=w\circ e_{\set S}$.
\item the function $m_{\set S}:\centralizer_\pauligroup\paren{\set S}\times \centralizer_\pauligroup\paren{\set S} \to \mathscr{N}$ is defined such that $m_{\set S}(l,l'):=\min \{\om_{\set S}(l),\om_{\set S}(l')\}$ --- that is, it gives the smaller of the weights of the smallest weight errors acting on respectively $\set L$ and $\set L'$;
\item the function $\vec M^{(N)}_{\set S}$ is defined such that $\vec M^{(N)}_{\set S}\paren{\vec P}$ is the sequence of $N$ integers such that $\vec M^{(N)}_{\set S}\paren{\vec P}_i := m_{\set S}\paren{\vec P_i}$;
\item the functions $p_1$ and $p_2$ are defined such that, given $(a,b):=x$, we have that $p_1(x):=a$ and $p_2(x):=b$.
\item the function $\set U:\powerset(\pauligroup\times\pauligroup)\to\powerset(\pauligroup)$ is defined (for convenience) such that $\set U\paren{\set X}:=\bigcup_{x\in\set X} \{p_1(x),p_2(x)\}$ --- that is, it `unpacks' a set of pairs of operators into a set of operators;  in an abuse of notation, we shall also allow $\set U$ to apply to sequences, so that $\set U(\lst P) := \set U\paren{\{\lst P_i\}_i}$, and to individual pairs, so that if $X$ is a single pair then $\set U(X) := \set U(\{X\})$;
\item finally, a \emph{choice of $N$ qubits stabilized by $\set S$}, $\lst P$, is a sequence of $N$ pairs of operators from the Pauli group such that 
\begin{enumerate}
\item $\set U(\lst P)\cap \set S = \emptyset$ --- that is, the operators in $\lst P$ are not included in $\set S$;
\item every pair in $\lst P$ is a conjugal pair with respect $\set S \cup \set U(\lst P)$.
\item and $\lst M^{(N)}(\lst P)$ is an ordered sequence.
\end{enumerate}
\end{itemize}

\end{definition}
%@nonl
%@-node:gcross.20091221145013.1277:Definition: Additional notation
%@+node:gcross.20091221145013.1278:Definition: Total ordering => Optimal choice of qubits
For every integer $N$, we can define a total ordering on the set of ordered sequences of $N$ integers as follows:  given two such sequences $\lst A$ and $\lst B$, $\lst A \le \lst B$ if and only if $\lst A_i \le \lst B_i$ for $i \in 1\dots N$.  Given this ordering and the notation above, we now precisely define what we mean by the ``best choice'' of logical qubits.

\begin{definition}
An \emph{optimal choice of $N$ qubits stabilized by $\set S$} is any maximizer of $\lst M^{(N)}$ over the set of all choices of $N$ qubits stabilized by $\set S$.
\end{definition}
%@-node:gcross.20091221145013.1278:Definition: Total ordering => Optimal choice of qubits
%@+node:gcross.20091221145013.1280:Big Proposition
%@+node:gcross.20091221145013.1281:Statement
Now that we have a precise notion of what it means to have made the best choice of logical qubits, it remains to show that there exists an algorithm that will make this choice.  The key to doing this resides in the following proposition:

\begin{proposition}
\label{proposition-optimality-condition}
Suppose we are given
\begin{itemize}
\item a choice of $N$ qubits, $\lst P$ stabilized by some set $\set S$;
\item a sequence $\lst E$ of $N$ Pauli operators such that for every $i\in\{1\dots N\}$, we have that
\begin{enumerate}
\item $\lst E_i$ is an undetectable error (with respect to $\set S$) acting on $p_1(\lst P_i)$;
\item $\lst E_i$ commutes with $p_2(\lst P_i)$;
\item $w(\lst E_i)=m(\lst P_i)$; and
\item $\lst E_i$ commutes with every operator in $\set U\paren{\{\lst P_j:j>i\}}$, i.e. with every operator in every pair after the $i$-th pair in the sequence.
\end{enumerate}
\item a sequence $\lst F$ of $N$ Pauli operators such that for every $i\in\{1\dots N\}$, we have that either $\lst F_i=I$ and $(\om_{\set S} \circ p_2)(\lst P_i) \ge w(\lst E_N)$, or else all of the following properties are satisfied
\begin{enumerate}
\item $\lst F_i$ is an undetectable error (with respect to $\set S$) acting on $p_2(\lst P_i)$;
\item $w\paren{\lst F_i}=(\om_{\set S} \circ p_2)(\lst P_i)$;
\item $\lst F_i$ commutes with every operator in $\set U\paren{\{\lst P_j:j>i\}}$, i.e. with every operator in every pair after the $i$-th pair in the sequence; and
\item $\lst F_i$ commutes with every operator in $\{p_2(\lst P_j): j<i\}$;
%\item and $w(\lst F_i)\le w(\lst E_N)$.
\end{enumerate}
\end{itemize}
Then $\lst P$ is an \emph{optimal choice of qubits stabilized by} $\set S$.
\end{proposition}
%@-node:gcross.20091221145013.1281:Statement
%@+node:gcross.20091221145013.1286:Definition: Generating set
For convenience, we define the following function:

\begin{definition}
Within the context of Proposition \ref{proposition-optimality-condition}, we define the function $\set G:(\genfun\circ\set U)(\lst P)\to(\powerset\circ\set U)(\lst P)$ such that $\set G(o)$ is the \emph{generating set} of $o$, that is the set such that $o=\prod_{g\in\set G(o)} g$.  For convenience, we shall abuse this notation by letting $\set G$ also apply to pairs of operators so that $\set G(X):=(\set G\circ p_1)(X)\cup(\set G\circ p_2)(X).$
\end{definition}
%@nonl
%@-node:gcross.20091221145013.1286:Definition: Generating set
%@+node:gcross.20091221145013.1282:Lemmas
In order to prove Proposition \ref{proposition-optimality-condition}, we shall need the following lemmas:
%@nonl
%@+node:gcross.20091221145013.1283:Combinations cannot make things worse
\begin{lemma}
\label{combinations-can't-make-things-worse}
Suppose we are given two operators $a,b\in\pauligroup$ such that $d_a:=\om_{\set S}(a)$ and $d_b:=\om_{\set S}(b)$.  Then $\om_{\set S}(a\cdot b) \ge \text{\min}\,(d_a,d_b)$.
\end{lemma}

\begin{proof}[Proof of Lemma]
Any undetectable error with respect to $\set S$ acting on $a\cdot b$ must also act on \emph{either} $a$ or $b$, since otherwise it cannot anti-commute with the product.  By assumption, there is no such error satisfying this property that has weight less than $\text{min}\,(d_a,d_b)$, and so $\om_{\set S}(a\cdot b) \ge \text{min}\,(d_a,d_b)$.
\end{proof}
%@-node:gcross.20091221145013.1283:Combinations cannot make things worse
%@+node:gcross.20091221145013.1284:Combinations cannot make things better
\begin{lemma}
\label{combinations-can't-make-things-better}
In the context of Proposition \ref{proposition-optimality-condition}, if we are given an operator $o\in (\genfun\circ\set U)(\lst P)$ such that $\om_{\set S}(o)\ge w(\lst E_n)$, then $\om_{\set S}(g)\ge w(\lst E_n)$ for every $g\in\set G(o)$.
\end{lemma}

\begin{proof}[Proof of Lemma]
Proof by contradiction.  Let $g\in \set G(o)$ be the operator such that $\om_{\set S}(g)<w(\lst E_n)$, the pair $\lst P_i$ to which $g$ belongs has minimal $i$, and if both operators in $\lst P_i$ meet these conditions then let $g$ be the first one.  Note that since $\om_{\set S}(g)<w(\lst E_n)$, by the assumptions of Proposition \ref{proposition-optimality-condition} we must have that either $g$ is the first member of a pair or $\lst F_i\ne I$.  In either case, we know that there is an operator --- respectively, either $\lst E_i$ or $\lst F_i$ --- which anti-commutes with $g$ but commutes with every other operator in $\set G(o)$.  Thus, the product of all of the operators in $\set G(o)$ must also anti-commute with $\lst E_i$, and thus $o$ anti-commutes with $\lst E_i$.  Since $w(\lst E_i)<w(\lst E_n)$, this contradicts the assumption that $\om_{\set S}(o)\ge w(\lst E_n)$, and so we conclude that all operators $g\in\set G(o)$ must satisfy $\om_{\set S}(g)\ge w(\lst E_n)$.
\end{proof}
%@-node:gcross.20091221145013.1284:Combinations cannot make things better
%@+node:gcross.20091226225139.1288:Deleting a qubits maintains the properties.
\begin{lemma}
\label{deleting-pair-maintains-properties}
In the context of Proposition \ref{proposition-optimality-condition}, for any $i\in\{1\dots N\}$, the sequences $\lst P'$, $\lst E'$ and $\lst F'$ defined by deleting the $i$-th element of respectively $\lst P$, $\lst E$ and $\lst F$ satisfy the same properties listed in the Proposition as their non-primed counterparts if we let $N:=N-1$.
\end{lemma}

\begin{proof}
Most of the properties look only at a single index in one or several sequences, and so the deletion of a different index does not affect them.  The only properties that would be affected by the deletion of an index are those which require that a Pauli operator commute with all of the operators contained in a subsequence, and this property is only made easier to satisfy by the deletion of an index.
\end{proof}
%@-node:gcross.20091226225139.1288:Deleting a qubits maintains the properties.
%@+node:gcross.20091230134131.1288:Single pair rearrangement.
\begin{lemma}
\label{single-pair-rearrangement}
In the context of Proposition \ref{proposition-optimality-condition}, let $A\in(\genfun\circ\set U)(\lst P)$ be a conjugal pair with respect to some set $\set R$, and $o$ be some Pauli operator such that $o\in\set G(A)$.  Then there exists an operator $B$ such that
\begin{enumerate}
\item $o\in p_1(B)$;
\item $o\notin p_2(B)$;
\item $B$ is a conjugal pair with respect to $\paren{\set R\backslash \set U(A)}\cup\set U(B)$; and
\item $(\genfun\circ\set U)(B) = (\genfun\circ\set U)(A)$.
\end{enumerate}
\end{lemma}

\begin{proof}
Let
$$
B:=
\begin{cases}
\paren{p_1(A),p_2(A)} & o\in \set G(p_1(A)), o\notin \set G(p_1(A)) \\
\paren{p_2(A),p_1(A)} & o\notin \set G(p_1(A)), o\in\set G(p_1(A)) \\
\paren{p_1(A),p_1(A)\cdot p_2(A)} & o\in\set G(p_1(A)), o\in\set G(p_1(A)) \\
\end{cases}
$$
Note that in any of the above cases, properties 1-2 are satisfied by construction, property 3 is satisfied because $p_1(B)$ and $p_2(B)$ are products of $p_1(A)$ and $p_2(A)$ which commute with every element in $\set R\backslash\{A\}$ and $\{p_1(B),p_2(B)\}=0$, and finally property 4 is satisfied because $\{p_1(A),p_2(A)\}\subseteq \genfun\paren{\{p_1(B),p_2(B)\}}$ and $\{p_1(B),p_2(B)\}\subseteq \genfun\paren{\{p_1(A),p_2(A)\}}$.
\end{proof}
%@nonl
%@-node:gcross.20091230134131.1288:Single pair rearrangement.
%@+node:gcross.20091226225139.1291:Directed Gaussian elimination
\begin{lemma}
\label{directed-gaussian-elimination-of-logicals}
In the context of Proposition \ref{proposition-optimality-condition}, suppose we are given
\begin{itemize}
\item a non-empty set of conjugal pairs, $\set C$, with respect to $\set U(\set C) \cup \set S$;
\item a set $\set A$ with the property that for any conjugal pair $X$ such that $\{p_1(X),p_2(X)\}\subseteq(\genfun\circ\set U)(\set C)$, we must have that $\set G(X) \cap \set A \ne \emptyset$; and
\item an element $a\in \set A$ with the property that there exists a pair $Y''\in\set C$ such that $a\in\set G(Y'')$
\end{itemize}
then there exists a conjugal pair $Y$ and set of conjugal pairs $\set D$, all with respect to $\set U\paren{\{Y\}\cup \set D} \cup \set S$, such that
\begin{enumerate}
\item $|\set D| = |\set C|-1$
\item $(\genfun\circ\set U)(\{Y\}\cup \set D)=(\genfun\circ\set U)(\set C)$;
\item $a\in (\set G\circ p_1)(Y)$ but $a\notin (\set G\circ p_2)(Y)$;
\item $a\notin \bigcup_{D\in \set D} \set G(D)$; and
\item for every conjugal pair $O\in\set D$, we have that $\set G(O) \cap \set A\backslash \{a\} \ne \emptyset$.
\end{enumerate}
\end{lemma}

\begin{proof}
Proof by induction on the size of $\set C$.  If $\set C=\{Y''\}$, then apply Lemma \ref{single-pair-rearrangement} letting $o:=a$, $A:=Y''$, and $Y:=B$, and we see that we have an operator $Y$ which is a conjugal pair with respect to $\{Y\}\cup\set S$ and also such that $(\genfun\circ\set U)(\{Y\})=(\genfun\circ\set U)(\{Y''\})$.  Let $\set D:=\emptyset$, and we see that the remaining properties hold trivially, so we are done.

Now let us assume that this lemma has been proven for the case where $|\set C|=n-1$, and we are given a set $\set C$ with $n$ elements.  Take any $X''\in\set C\backslash\{Y''\}$, and apply the lemma to $\set C\backslash \{X''\}$, $\set A$, $a$ and $Y''$ to obtain the objects $Y'$ and $\set D'$ described in this Lemma without the primes.  If $a\notin\set G(X'')$, then by the assumptions of the Lemma we know that $\set G(X'') \cap \set A\backslash \{a\} \ne \emptyset$, so let $Y:=Y'$ and $\set D:=\set D'\cup\{X''\}$, and we are done.

Otherwise, apply Lemma \ref{single-pair-rearrangement}, setting $A:=X''$, $o:=a$, and $X':=B$, and let $X:=\paren{p_1(X')\cdot p_1(Y'),p_2(X')}$ and $Y:=\paren{p_1(Y'),p_2(Y')\cdot p_2(X')}$.  Note $X'$ and $Y'$ are conjugal pairs with respect to $\set U\paren{\{X',Y'\}\cup\set D}\cup\set S$ and $\{X',Y'\}\cap \paren{\set D\cup\set S}=\emptyset$, and so by Lemma \ref{combining-pairs} we conclude that $X$ and $Y$ are conjugal pairs with respect to $\set U\paren{\{X,Y\}\cup\set D}\cup\set S$, and also that $(\genfun\circ\set U)(\{X,Y\})=(\genfun\circ\set U)(\{X',Y'\})$;  since $X'$ was obtained from applying Lemma \ref{single-pair-rearrangement} to $X''$ and $a$, we furthermore conclude that $(\genfun\circ\set U)(\{X,Y\})=(\genfun\circ\set U)(\{X'',Y'\})$.  Since $X'$ was obtained as a result of Lemma \ref{single-pair-rearrangement}, we know that $a\in (\set G \circ p_1)(X')$ but $a\notin (\set G \circ p_2)(X')$, and we also know from the earlier recursive application of this Lemma that $a\in (\set G \circ p_1)(Y')$ but $a\notin (\set G \circ p_2)(Y')$.  Thus, we observe that by construction, $a\in (\set G \circ p_1)(Y)$, and $a\notin \paren{(\set G \circ p_2)(Y) \cup \set G(X)}$.

Let $\set D:=\{X\}\cup\set D'$, and observe that $|\set D|=|\set D'|+1=|\set C\backslash \{X''\}|-1+1=|\set C|-1$, and also that $(\genfun\circ\set U)(\{Y\}\cup\set D)=(\genfun\circ\set U)(\{X,Y\}\cup\set D')=(\genfun\circ\set U)(\{X''\}\cup(\{Y'\}\cup\set D'))=(\genfun\circ\set U)(\{X''\}\cup(\set C'\backslash\{X''\}))=(\genfun\circ\set U)(\set C)$.  Furthermore, by the earlier recursive application of this Lemma we know that $a\notin\set G(O)$ for every $O\in \set D'$, so since we have also established that $a\notin \set G(X)$, we conclude that $a\notin\set G(O)$ for every $O\in \set D$;  since also know that every such $O$ must also satisfy $\set G(O)\cap \set A \ne \emptyset$, we conclude that every such $O$ satisfies $\set G(O) \cap A\backslash\{a\}\ne\emptyset$.
\end{proof}
%@+node:gcross.20100101105725.1294:Undirected Gaussian elimination
\begin{corolary}[of Lemma \ref{directed-gaussian-elimination-of-logicals}]
\label{undirected-gaussian-elimination-of-logicals}
In the context of Proposition \ref{proposition-optimality-condition}, suppose we are given the sets $\set C$ and $\set A$ described in Lemma \ref{directed-gaussian-elimination-of-logicals};  then there exists the objects $a$, $Y$, and $\set D$ described in Lemma \ref{directed-gaussian-elimination-of-logicals}.
\end{corolary}

\begin{proof}
Take any pair $Y'\in\set C$.  By the assumptions of this Lemma, we know that $\set G(Y')\cap \set A \ne \emptyset$, which implies that there exists an element $a\in A$ such that either $a\in (\set G\circ p_1)(Y')$ or $a\in (\set G\circ p_2)(Y')$.  The existance of $Y$ and $\set D$ then follow immediately from the application of Lemma \ref{directed-gaussian-elimination-of-logicals}.
\end{proof}
%@nonl
%@-node:gcross.20100101105725.1294:Undirected Gaussian elimination
%@+node:gcross.20091230212318.1289:Elimination to create subset
\begin{corolary}[of Lemma \ref{directed-gaussian-elimination-of-logicals}]
\label{elimination-to-create-subset}
In the context of Proposition \ref{proposition-optimality-condition}, suppose we are given the sets $\set C$ and $\set A$ described in Lemma \ref{directed-gaussian-elimination-of-logicals};  then there exists a set of conjugal pairs, $\set X$, with respect to $\set X\cup\set S$, and a subset of operators, $\set O\subseteq \set A$, such that
\begin{enumerate}
\item $|\set X|=|\set O|=|\set C|$;
\item $(\genfun\circ\set U)(\set X)=(\genfun\circ\set U)(\set C)$; and
\item for every $o\in\set O$, there is a conjugal pair $Y\in\set X$ such that $o\in\set G(Y)$;
\end{enumerate}
\end{corolary}

\begin{proof}
Proof by induction.  If $\set C$ is empty, then the empty sets trivially satisfy this Corolary.

Now suppose that we have proven this Corolary for $|\set C|=N-1$, and assume we have been given sets $\set C$ and $\set A$ such that $|\set C|=N$.  Applying Corolary \ref{undirected-gaussian-elimination-of-logicals} to $\set C$ and $\set A$ we obtain the conjugal pair $Y$,  the set of conjugal pairs $\set D$, and the element $a$ described in the conclusions of that Corolary.  Apply this Corolary recursively to the respective sets $\set D$ and $\set A\backslash\{a\}$, we obtain the sets $\set X'$ and $\set O'$ described (without the primes) in this Corolary; let $\set X := \set X'\cup\{Y\}$ and $\set O:=\set O'\cup\{a\}$.  Note that $\set X$ is a set of conjugal pairs with respect to $\set X\cup\set S$ since $\set X'$ is a set of conjugal pairs with respect to $\set X'\cup\set S$, and we know that the operators in $Y$ commute with every operator in every pair in $\set X'$ since they commute with every operator in $(\genfun\circ\set U)(\set D)=(\genfun\circ\set U)(\set X')$.

First, observe that $|\set O|=|\set O'|+1$ since $a\notin \set O'$.  Futhermore, $a\notin \bigcup_{x\in \set U(\set X')} \set G(x)$ since $a\notin \bigcup_{x\in \set U(\set D)} \set G(x)$ by Corolary \ref{undirected-gaussian-elimination-of-logicals} and $(\genfun\circ\set U)(\set D)=(\genfun\circ\set U)(\set X')$ by recursive application of this Corolary.  Thus, $|\set X|=|\set X'|+1$ since $Y\notin X'$ as $a\in\set G(Y)$, and $|\set X|=|\set O|=|\set D|+1=|\set C|-1+1=|\set C|$.

Second, observe that since $(\genfun\circ\set U)(\set X')=(\genfun\circ\set U)(\set D)=(\genfun\circ\set U)(\set C)$ by recursive application of this Corolary and $(\genfun\circ\set U)(\{Y\}\cup\set D)=(\genfun\circ\set U)(\set C)$ by Corolary \ref{undirected-gaussian-elimination-of-logicals}, we conclude that $(\genfun\circ\set U)(\set X) = (\genfun\circ\set U)(\{Y\}\cup\set X') = (\genfun\circ\set U)(\{Y\}\cup\set D) = (\genfun\circ\set U)(\set C)$.

Finally, observe that for every $o\in\set O$ we either have that $o=a$, in which case $Y\in\set X$ and $o\in\set G(Y)$, or $o\in \set A\backslash\{a\}$, in which case by recursive application of this Corolary we know that there is an operator $Z\in \set X'\subseteq \set X$ such that $o\in\set G(Z)$.
\end{proof}
%@nonl
%@-node:gcross.20091230212318.1289:Elimination to create subset
%@+node:gcross.20091230212318.1290:Upper bound on number of pairs
\begin{corolary}[of Lemma \ref{directed-gaussian-elimination-of-logicals}]
In the context of Proposition \ref{proposition-optimality-condition}, there are no sets $\set C$ and $\set A$ as described in Lemma \ref{directed-gaussian-elimination-of-logicals} such that $|\set C|>|\set A|$.
\end{corolary}

\begin{proof}
Proof by contradiction.  By Corolary \ref{elimination-to-create-subset}, there would have to exist a subset $\set O\subseteq\set A$ such that $|\set C|=|\set O|>|\set A|$, which is impossible.
\end{proof}
%@-node:gcross.20091230212318.1290:Upper bound on number of pairs
%@+node:gcross.20091230212318.1291:All are present
\begin{corolary}[of Lemma \ref{directed-gaussian-elimination-of-logicals}]
\label{all-are-present}
In the context of Proposition \ref{proposition-optimality-condition}, given $\set C$ and $\set A$ as described in Lemma \ref{directed-gaussian-elimination-of-logicals} where $|\set C|=|\set A|$, for every element $a\in\set A$ there exists a pair $Y\in\set C$ such that $a\in\set G(Y)$.
\end{corolary}

\begin{proof}
Applying Corolary \ref{elimination-to-create-subset} and the fact that the only subset of $\set A$ the same size of $\set A$ is $\set A$ itself, we see that for every $a$ there exists an operator $o\in(\genfun\circ\set U)(\set C)$ such that $a\in\set G(o)$, which implies that there must be some pair $Y\in\set C$ such that $a\in\set G(Y)$.
\end{proof}
%@-node:gcross.20091230212318.1291:All are present
%@+node:gcross.20100101105725.1298:Elimination to isolate
\begin{corolary}[of Lemma \ref{directed-gaussian-elimination-of-logicals}]
\label{elimination-to-isolate}
In the context of Proposition \ref{proposition-optimality-condition}, suppose we are given non-empty sets of conjugal pairs, $\set B$ with respect to $\set B\cup\set S$, and $\set C$ with respect to $\set U(\set C)\cup\set S$, such that $|\set B|=|\set C|$ and for every pair $Y''\in(\genfun\circ\set U)(\set C)$ there exists a pair $(a,b)\in\set B$ such that either $a\in (\set G\circ p_1)(Y'')$ and $b\in (\set G\circ p_2)(Y'')$ or $a\in (\set G\circ p_2)(Y'')$ and $b\in (\set G\circ p_1)(Y'')$, and we don't have $\{a,b\}\cap(\set G\circ p_1)(Y'')=\{a,b\}\cap(\set G\circ p_2)(Y'')=\{a,b\}$.  Then for every $(a,b)\in\set B$ there exists a conjugal pair, $Y$, and a set of conjugal pairs, $\set D$, all with respect to $\set U\paren{\{Y\}\cup\set D}\cup\set S$, such that
\begin{itemize}
\item $|\{Y\}\cup \set D| = |\set C|$
\item $(\genfun\circ\set U)(\{Y\}\cup \set D)=(\genfun\circ\set U)(\set C)$;
\item $\{p_1(O):O\in\set B\}\cap (\set G\circ p_1)(Y)=\{a\}$;
\item $\{p_1(O):O\in\set B\}\cap (\set G\circ p_2)(Y)=\emptyset$;
\item $b\in (\set G\circ p_2)(Y)$; and
\item $b\notin (\set G\circ p_1)(Y)$;
\end{itemize}
\end{corolary}

\begin{proof}
Proof by induction.  First suppose that $\set B=\{(a,b)\}$ and $\set C=\{A\}$.  Apply Lemma \ref{single-pair-rearrangement} to $A$ and $o:=a$, and let $Y':=B$ be the operator described in the conclusion.  If $b\in (\set G\circ p_1)(Y')$ then let $Y:=(p_1(Y')\cdot p_2(Y'),p_2(Y')$, otherwise let $Y:=Y'$;  note that by construction and the conclusions of Lemma \ref{single-pair-rearrangement} we know that $a\in (\set G \circ p_1)(Y)$, $a\notin (\set G\circ p_2)(Y)$, $b\notin (\set G \circ p_1)(Y)$, and $b\in (\set G \circ p_2)(Y)$.  Furthermore observe that since the members of $Y'$ can be recovered from products of the members of $Y$, by Lemma \ref{single-pair-rearrangement} we see that $(\genfun\circ\set U)(\{Y\})=(\genfun\circ\set U)(A)$.  Let $\set D:=\emptyset$, and we see trivially that $|\{Y\}\cup \set D| = |\set C|$.

Now suppose that we have proven this Lemma for $|\set B|=N-1$, and assume we have been given sets $\set B$ and $\set C$ such that $|\set B|=N>1$.  Let $\set A:=\bigcup_{O\in\set B} p_1(O)$, i.e. the set of first members of pairs in $\set B$.  By the assumptions of this Lemma, we know that for every $O\in\set C$ we have that $\set G(O) \cap \set A \ne \emptyset$.  Thus, by Corolary \ref{all-are-present} and Lemma \ref{directed-gaussian-elimination-of-logicals}, we know that we may pick arbitrarily some pair $(c,d)\in\set B\backslash\{(a,b)\}$ and apply Lemma \ref{directed-gaussian-elimination-of-logicals} to $\set A$, $\set C$, and $a:=c$ to obtain the objects $X:=Y$ and $\set C':=\set D$ described in Lemma \ref{directed-gaussian-elimination-of-logicals}.  Note furthermore that a consequence of the assumptions of this Corolary combined with the conclusions of Lemma \ref{directed-gaussian-elimination-of-logicals} is that for every pair $Y''\in(\genfun\circ\set U)(\set C')\subseteq(\genfun\circ\set U)(\set C)$ there must exist a pair $(x,y)\in\set B\backslash\{(c,d)\}$ such that either $x\in (\set G\circ p_1)(Y'')$ and $y\in (\set G\circ p_2)(Y'')$ or $x\in (\set G\circ p_2)(Y'')$ and $y\in (\set G\circ p_1)(Y'')$, and we don't have $\{x,y\}\cap(\set G\circ p_1)(Y'')=\{x,y\}\cap(\set G\circ p_2)(Y'')=\{x,y\}$.  Thus, invoking the inductive hypothesis, we recursively apply this Corolary to the sets $\set B:=\set B\backslash\{(c,d)\}$ and $\set C:=\set C'$, obtaining the objects $Y$ and $\set D'$ described (without the primes) in this Corolary.

Let $\set D:=\{X\}\cup\set D'$.  We know that $X$ is a conjugal pair with respect to $\{X\}\cup\set C'\cup\set S$, and we also know that $(\genfun\circ\set U)(\{Y\}\cup\set D')=(\genfun\circ\set U)(\set C')$;  therefore, we know that the operators in $X$ commute with the operators in $Y$ and also the operators in $(\genfun\circ\set U)(\set D')\subset(\genfun\circ\set U)(\set C')$.  Thus, $\{X\}\cup\{Y\}\cup\set D'$ is a set of conjugal pair with respect to $\{Y\}\cup\set D\cup\set S$.   Also, since in particular $X\notin\set D'$, we see that  $|\set D|=|\set D'|+1=|\set C|-1$ and thus $|\{Y\}\cup\set D|=|\set C|$.  Furthermore, we know that $(\genfun\circ\set U)(\{Y\}\cup\set D)=(\genfun\circ\set U)(\{Y\}\cup\{X\}\cup\set D')(\genfun\circ\set U)(\{X\}\cup\set C')=(\genfun\circ\set U)(\set C)$.

By recursive application of this Corolary we know that
$$\{p_1(O):O\in\set B\backslash\{(c,d)\}\}\cap (\set G\circ p_1)(Y)=\{a\},$$
$$\{p_1(O):O\in\set B\backslash\{(c,d)\}\}\cap (\set G\circ p_2)(Y)=\emptyset,$$
$b\in (\set G \circ p_2)(Y)$ and $b\notin (\set G\circ p_1)(Y)$.  By Lemma \ref{directed-gaussian-elimination-of-logicals} we know that $c\notin \bigcup_{b\in(\genfun\circ\set U)(\set B)} \set G(b)$, and thus 
$$\{p_1(O):O\in\set B\}\cap (\set G\circ p_1)(Y)=\{a\}$$
and
$$\{p_1(O):O\in\set B\}\cap (\set G\circ p_2)(Y)=\emptyset.$$
\end{proof}
%@-node:gcross.20100101105725.1298:Elimination to isolate
%@-node:gcross.20091226225139.1291:Directed Gaussian elimination
%@+node:gcross.20100101105725.1299:Sequence rearrangement
\begin{lemma}
In the context of Proposition \ref{proposition-optimality-condition}, suppose we are given an integer $1\le k\le N$ and a conjugal pair with respect to $\set S$, $(a,b):=O\in(\genfun\circ\set U)(\lst P)$, such that
\begin{itemize}
\item $m(O)\ge m(\lst P_n)$;
\item $\{p_1(\lst P_i)\}_i\cap (\set G\circ p_1)(O)=\{p_1(\lst P_n)\}$;
\item $\{p_1(\lst P_i)\}_i\cap (\set G\circ p_2)(O)=\emptyset$;
\item $p_2(\lst P_n)\notin (\set G\circ p_1)(O)$;
\item $p_2(\lst P_n)\in (\set G\circ p_2)(O)$; and
\item $\{p_2(\lst P_i):i<k\}\cap \set G(\set O)=\emptyset$.
\end{itemize}
Then there exists a choice of qubits, $\lst P'$, stabilized by $\set S$, such that
\begin{enumerate}
\item $\lst P_N=O$;
\item $(\genfun\circ\set U)(\lst P)=(\genfun\circ\set U)(\lst P')$;
\item the first $N-1$ elements of $\lst P'$, $\lst E$ and $\lst F$ satisfy the properties listed in Proposition \ref{proposition-optimality-condition};
\item $\lst M^{(N)}(\lst P')=\lst M^{(N)}(\lst P)$; and
\item $\lst P_i=\lst P'_i$ for $i<k$.
\end{enumerate}
\end{lemma}

\begin{proof}
Proof by reverse induction on the $k$.  For the base case, suppose that $k=N$; then we are done setting $\lst P''=\lst P$ and employing the result of Lemma \ref{deleting-pair-maintains-properties}.

Now suppose we have proven this Lemma for all integers $k+1$ and larger and we are given an integer $k$ and a conjugal pair $O$ satisfying the assumptions of this Lemma.  For notational convenience, let $(c,d):=\lst P_k$.
Define
$$
a' :=
\begin{cases}
a\cdot d & d\in\set G(a) \\
a & \text{otherwise} \\
\end{cases},
\quad
b' :=
\begin{cases}
b\cdot d & d\in\set G(b) \\
b & \text{otherwise} \\
\end{cases}
$$
and $O':=(a',b')$.  By Lemma \ref{combinations-can't-make-things-better}, we know that since $m(O)=m(\lst P_n)$ and $d\in\set G(O)$, we have that $\om_{\set S}(d)\ge m(O)$ and thus $m(O')\ge m(O)$.  Note that $O'$ satisfies the assumptions of this Lemma with the integer $k+1$, and so we apply this Lemma recursively to obtain the choice of qubits $\lst P'':=\lst P'$ described in this Lemma;  note in particular that $\lst P''_k=\lst P_k$.

Let 
$$
c' :=
\begin{cases}
c\cdot y & d\in\set G(a) \\
c & \text{otherwise}
\end{cases},
\,\, \text{and} \,\,
x' :=
\begin{cases}
x\cdot d & d\in\set G(a) \\
x & \text{otherwise}
\end{cases}.
$$
By Lemma \ref{combining-pairs} we know that $(c',d)$ and $(x',y)$ are conjugal pairs with respect to $(\{\lst P''_i\}_i\backslash \{(c,d),(x,y)\})\cup\{(c',d),(x',y)\}\cup\set S$ and also that $\genfun\paren{\{c,d,x,y\}}=\genfun\paren{\{c',d,x',y\}}$.

Next let
$$
c'' :=
\begin{cases}
c'\cdot x' & d\in\set G(b) \\
c' & \text{otherwise}
\end{cases},
\,\, \text{and} \,\,
y' :=
\begin{cases}
y\cdot d & d\in\set G(b) \\
y & \text{otherwise}
\end{cases}.
$$
By Lemma \ref{combining-pairs} we know that $(c'',d)$ and $(x',y')$ are conjugal pairs with respect to $(\{\lst P''_i\}_i\backslash \{(c,d),(x,y)\})\cup\{(c'',d),(x',y')\}\cup\set S$ and also that $\genfun\paren{\{c,d,x,y\}}=\genfun\paren{\{c',d,x',y\}}=\genfun\paren{\{c'',d,x',y'\}}$.  Observe that by construction, $(x',y')=O$.

Define
$$
\lst P':=
\begin{cases}
\lst P''_i & i < k \\
(c'',d) & i = k \\
\lst P''_i & k < i < N \\
(x',y')\equiv O & k = N.
\end{cases}
$$
First observe that $\set G(O')\subseteq (\genfun\circ\set U)(\{\lst P_i:i>k\})$, and hence every operator in $\set G(O')$ must commute with $\lst E_k$ by the assumptions of Proposition \ref{proposition-optimality-condition}.  Thus, $x$, $x'$ (which is $x$ times possibly a factor of $d$, which commutes with $\lst E_k$) and $y$ all commute with $\lst E_k$, and hence $c''$ anti-commutes with $\lst E_k$.  Furthermore, we also know that since $m(O')\ge m(O) \ge m(\lst P_n)$ that therefore $\om_{\set S}(x')\ge m(\lst P_n)$ and $\om_{\set S}(x')\ge m(\lst P_n)$, so by Lemma \ref{combinations-can't-make-things-worse} we know that $\om_{\set S}(c'') \ge \text{\min}\,\paren{\om_{\set S}(c),\om_{\set S}(x'),\om_{\set S}(y)}.$  Thus, $m(\lst P'_k)=m(\lst P''_k)=m(\lst P_k)$, and since $m(O)=m(\lst P'_N)=m(\lst P''_N)=m(\lst P_N)$, and so $M(\lst P')=M(\lst P)$.

Now note that the only difference between $\lst P'$ and $\lst P''$ are the pairs at indices $k$ and $N$.  This, combined with the result from the recursive application of this Lemma used to obtain $\lst P''$, implies that $\lst P_i=\lst P'_i$ for $i < k$.  Furthermore, the transformation used to obtain the new pairs from the old preserves the set of operators generated by the pairs, and also guarantees that they are conjugal pairs with respect to $\set U(\lst P')\cup\set S$.  Thus, since the fact that $M(\lst P')=M(\lst P)$ implies that $M(\lst P')$ is ordered, we conclude that $\lst P'$ is a choice of qubits stabilized by $\set S$ such that $(\genfun\circ\set U)(\lst P)=(\genfun\circ\set U)(\lst P')$.

Finally, recall that $c''$ anti-commutes with $\lst E_i$, and this fact combined with the result from the recursive application of this Lemma used to obtain $\lst P''$ and the fact that only the pair at index $k$ was changed out of the first $N-1$ pairs in the sequence implies that the first $N-1$ elements of $\lst P'$, $\lst E$, and $\lst F$ satisfy the properties listed in Proposition \ref{proposition-optimality-condition}.
\end{proof}
%@-node:gcross.20100101105725.1299:Sequence rearrangement
%@-node:gcross.20091221145013.1282:Lemmas
%@+node:gcross.20091221145013.1285:Proof
\begin{proof}[Proof of Proposition \ref{proposition-optimality-condition}]
We shall prove this proposition by using induction on the length of the sequences $\lst P$ and $\lst E$.  For the base case, note that all of the properties hold vacuously for the empty sequence.

Now suppose that we have proven that this proposition holds for sequences of length $N-1$, and have been given sequences of length $N$.  We now need to show that given any other choice of $N$ qubits stabilized by $\set S$, $\lst P'$, such that $\genfun\paren{\set U\paren{\{\lst P_i\}_i}}=\genfun\paren{\set U\paren{\{\lst P'_i\}_i}}$, we have that $\lst M(\lst P') \le \lst M(\lst P)$.

Consider the set $\set O' := \{\lst P'_i:m(\lst P'_i)=\max_i m(\lst P'_i) \}$.  Since each operator in each pair in this set can be expressed as a product of elements of $\set U{\paren{\{\lst P_i\}_i}},$ and the the two operators in each pair in $\set O'$ must anti-commute with each other, we conclude that there must be a pair $x\in\set O'$ such that for some $i\in\{1\dots N\}$ and $k\in \{1,2\}$ we have that $p_1(\lst P_i)\in (\set G \circ p_k)(x)$ and $p_2(\lst P_i)\in (\set G \circ p_{3-k})(x)$ --- in other words, for some pair $\lst P_i$ we must have that the generating set of one of the members of $x$ contains one of the members of $\lst P_i$ and the generating set of the other member of $x$ contains the other.  Without loss of generality, assume that $x=\lst P'_n$, since reordering any subsequence of pairs over which the function $m$ is constant does not affect the value $\lst M(\lst P')$.  Furthermore without loss of generality, assume that $k=1$ --- that is, for some $i$ we have that $p_1(\lst P_i)\in(\set G \circ p_1)(x)$ and $p_2(\lst P_i)\in(\set G \circ p_2)(x)$ --- since flipping the members of a pair likewise does not affect the value of $\lst M(O')$.

\end{proof}
%@-node:gcross.20091221145013.1285:Proof
%@-node:gcross.20091221145013.1280:Big Proposition
%@-others
%@-node:gcross.20090511123440.3:Optimal generators
%@-node:gcross.20090423002455.2:Algorithm
%@-others

\end{document}
%@nonl
%@-node:gcross.20090405101642.3:@thin CodeQuest.tex
%@-leo
