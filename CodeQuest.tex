%@+leo-ver=4-thin
%@+node:gcross.20090405101642.3:@thin CodeQuest.tex
%@@language latex

%@<< Prelude >>
%@+node:gcross.20090405101642.4:<< Prelude >>
\documentclass[twocolumn,showpacs,preprintnumbers,amsmath,amssymb,nofootinbib,pra,floatfix]{revtex4-1}

\usepackage{mathrsfs,amsthm,clrscode,comment}

\newtheorem{theorem}{Theorem}
\newtheorem{proposition}{Proposition}
\newtheorem{lemma}{Lemma}
\newtheorem{corolary}{Corolary}

\newenvironment{definition}[1][Definition]{\begin{trivlist}
\item[\hskip \labelsep {\bfseries #1}]}{\end{trivlist}}

\newenvironment{example}[1][Example]{\begin{trivlist}
\item[\hskip \labelsep {\bfseries #1}]}{\end{trivlist}}
\newenvironment{remark}[1][Remark]{\begin{trivlist}
\item[\hskip \labelsep {\bfseries #1}]}{\end{trivlist}}

%@-node:gcross.20090405101642.4:<< Prelude >>
%@nl

%@<< Macros >>
%@+node:gcross.20091029172439.1256:<< Macros >>
\newcommand{\lst}{\vec}
\newcommand{\set}{\tilde}

\newcommand{\genfun}{\mathcal{G}}
\newcommand{\pauligroup}{\mathfrak{P}}
\newcommand{\powerset}{\mathcal{P}}
\newcommand{\centralizer}{\mathcal{C}}

%@+leo-ver=4
%@+node:@file /Users/cog/Documents/macros.tex
%@@language latex

\newcommand{\drv}[2]{ \frac{d#1}{d#2} }
\newcommand{\pdrv}[2]{ \frac{\partial #1}{\partial #2} }
\newcommand{\dpdrv}[2]{ \frac{\partial^2 #1}{\partial #2^2} }
\newcommand{\ddrv}[2]{\frac{d^2#1}{d#2^2}}
\newcommand{\ndrv}[3]{\frac{d^{#3}#1}{d#2^{#3}}}
\newcommand{\half}{\frac{1}{2}}
\newcommand{\quarter}{\frac{1}{4}}

\newcommand{\intii}{\int^{\infty}_{-\infty}}

\newcommand{\eqn}[2]{\begin{equation}\label{#1}#2\end{equation}}

\newcommand{\E}[1]{\times 10^{#1}}

\newcommand{\om}{\omega}
\newcommand{\bomega}{\vec{\omega}}
\newcommand{\bom}{\bomega}
\newcommand{\domega}{\dot\omega}
\newcommand{\dom}{\domega}

%\newcommand{\th}{\theta}

\newcommand{\unit}[1]{\,\text{#1}}

\newcommand{\g}{\unit{g}}
\newcommand{\kg}{\unit{kg}}
\newcommand{\m}{\unit{m}}
\newcommand{\um}{\,\mu\!\unit{m}}
\newcommand{\micrometer}{\um}
\newcommand{\mm}{\unit{mm}}
\newcommand{\nm}{\unit{nm}}
\newcommand{\picom}{\unit{pm}}
\newcommand{\km}{\unit{km}}
\newcommand{\cm}{\unit{cm}}
\newcommand{\s}{\unit{s}}
\newcommand{\us}{\,\mu\!\unit{s}}
\newcommand{\ns}{\unit{ns}}
\renewcommand{\min}{\unit{min}}
\newcommand{\N}{\unit{N}}
\newcommand{\Li}{\unit{L}}
\newcommand{\Hz}{\unit{Hz}}
\newcommand{\mHz}{\unit{mHz}}
\newcommand{\kHz}{\unit{kHz}}
\newcommand{\MHz}{\unit{MHz}}
\newcommand{\GHz}{\unit{GHz}}
\newcommand{\J}{\unit{J}}
\newcommand{\kJ}{\unit{kJ}}
\newcommand{\mol}{\unit{mol}}
\newcommand{\K}{\unit{K}}
\newcommand{\W}{\unit{W}}
\newcommand{\kW}{\unit{kW}}
\newcommand{\V}{\unit{V}}
\newcommand{\eV}{\unit{eV}}
\newcommand{\keV}{\unit{keV}}
\newcommand{\MeV}{\unit{MeV}}
\newcommand{\meV}{\unit{meV}}
\newcommand{\ueV}{\unit{ueV}}
\newcommand{\T}{\unit{T}}
\newcommand{\C}{\unit{C}}
\newcommand{\hour}{\unit{hr}}
\newcommand{\dayunit}{\unit{day}}

\newcommand{\tr}{\text{tr}\,}

\newcommand{\vel}{\,\frac{\m}{\s}}

\newcommand{\speedoflight}{3\E{8}\vel}


\renewcommand{\max}{\text{max}}

\renewcommand{\dim}[1]{\left[#1\right]}

\newcommand{\rms}{\text{rms}}

\renewcommand{\deg}{^\circ}

\newcommand{\dx}{\dot x}
\newcommand{\dv}{\dot v}
\newcommand{\ddx}{\ddot x}

\newcommand{\mathboit}{\textbf}

\newcommand{\hH}{\hat H}
\newcommand{\ha}{\hat a}
\newcommand{\hadag}{\hat a^\dagger}

\newcommand{\btheta}{\theta}
\newcommand{\bF}{\mathboit F}
\newcommand{\bL}{\mathboit L}
\newcommand{\br}{\mathboit r}
\newcommand{\bl}{\mathboit l}
\newcommand{\bp}{\mathboit p}
\newcommand{\bc}{\mathboit c}
\newcommand{\bv}{\mathboit v}
\newcommand{\bA}{\mathboit{A}}
\newcommand{\bB}{\mathboit{B}}
\newcommand{\bJ}{\mathboit{J}}
\newcommand{\bD}{\mathboit{D}}
\newcommand{\bS}{\mathboit{S}}
\newcommand{\bT}{\mathboit{T}}
\newcommand{\bI}{\mathboit{I}}
\newcommand{\bR}{\mathboit{R}}
\newcommand{\bK}{\mathboit{K}}
\newcommand{\bH}{\mathboit{H}}
\newcommand{\bM}{\mathboit{M}}
\newcommand{\bG}{\mathboit{G}}
\newcommand{\bP}{\mathboit{P}}
\newcommand{\bE}{\mathboit{E}}
\newcommand{\bV}{\mathboit{V}}
\newcommand{\bQ}{\mathboit{Q}}
\newcommand{\bO}{\mathboit{O}}
\newcommand{\bX}{\mathboit{X}}
\newcommand{\bZ}{\mathboit{Z}}
\newcommand{\bC}{\mathboit{C}}
\newcommand{\bN}{\mathboit{N}}

\newcommand{\bx}{\mathboit x}
\newcommand{\bz}{\mathboit z}
\newcommand{\be}{\mathboit e}
\newcommand{\bg}{\mathboit g}

\newcommand{\bn}{\mathboit n}
\newcommand{\ba}{\mathboit a}
\newcommand{\bb}{\mathboit b}
\newcommand{\bu}{\mathboit u}
\newcommand{\bs}{\mathboit s}

\newcommand{\bi}{\mathboit i}
\newcommand{\bj}{\mathboit j}
\newcommand{\bk}{\mathboit k}

\newcommand{\mO}{\mathscr O}
\newcommand{\mL}{\mathscr L}
\newcommand{\mH}{\mathscr H}

\newcommand{\hx}{\hat{\mathboit x}}
\newcommand{\hy}{\hat{\mathboit y}}
\newcommand{\hz}{\hat{\mathboit z}}
\newcommand{\hr}{\hat{\mathboit r}}
\newcommand{\he}{\hat{\mathboit e}}
\newcommand{\htheta}{\hat{\theta}}

\newcommand{\hse}{\,\hat{\mathboit e}\,}

\newcommand{\va}{\vec{a}}
\newcommand{\vb}{\vec{b}}
\newcommand{\vc}{\vec{c}}
\newcommand{\vd}{\vec{d}}
%\newcommand{\vr}{\vec{r}}
\newcommand{\vx}{\vec{x}}
\newcommand{\vy}{\vec{y}}
\newcommand{\vz}{\vec{z}}
%\newcommand{\vr}{\vec{r}}
\newcommand{\vp}{\vec{p}}
\newcommand{\vq}{\vec{q}}
\newcommand{\vk}{\vec{k}}
\newcommand{\vv}{\vec{v}}
\newcommand{\vu}{\vec{u}}
\newcommand{\vf}{\vec{f}}
\newcommand{\vR}{\vec{R}}
\newcommand{\vP}{\vec{P}}
\newcommand{\vS}{\vec{S}}
\newcommand{\vs}{\vec{s}}
\newcommand{\vL}{\vec{L}}
\newcommand{\vl}{\vec{l}}
\newcommand{\vJ}{\vec{J}}
\newcommand{\vj}{\vec{j}}
\newcommand{\vF}{\vec{F}}
\newcommand{\vE}{\vec{E}}
\newcommand{\vB}{\vec{B}}
\newcommand{\vM}{\vec{M}}
\newcommand{\vA}{\vec{A}}
\newcommand{\vH}{\vec{H}}
\newcommand{\vD}{\vec{D}}
\newcommand{\vsigma}{\vec{\sigma}}
\newcommand{\valpha}{\vec{\alpha}}

\newcommand{\vepsilon}{\vec{\epsilon}}

\newcommand{\dtbx}{\dot \bx}
\newcommand{\ddtbx}{\ddot \bx}
\newcommand{\dtbr}{\dot \br}
\newcommand{\ddtbr}{\ddot \br}
\newcommand{\dtbR}{\dot \bR}
\newcommand{\ddtbR}{\ddot \bR}

\newcommand{\dtx}{\dot x}
\newcommand{\dty}{\dot y}
\newcommand{\dtz}{\dot z}
\newcommand{\dts}{\dot s}
\newcommand{\dtheta}{\dot \theta}
\newcommand{\dttheta}{\dot \theta}
\newcommand{\ddtx}{\ddot x}
\newcommand{\ddty}{\ddot y}
\newcommand{\ddtz}{\ddot z}
\newcommand{\ddts}{\ddot s}
\newcommand{\ddttheta}{\ddot \theta}
\newcommand{\dtr}{\dot r}
\newcommand{\ddtr}{\ddot r}
\newcommand{\dtl}{\dot l}
\newcommand{\ddtl}{\ddot l}

\newcommand{\atan}{\text{atan}\,}
\newcommand{\atanh}{\text{atanh}\,}
\newcommand{\acot}{\text{acot}\,}
\newcommand{\acoth}{\text{acoth}\,}

%\newcommand{\br}{\mathboit r}
\renewcommand{\r}{\br}
%\newcommand{\bv}{\mathboit v}
\renewcommand{\v}{\bv}

\newcommand{\curl}{\nabla\times}

\newcommand{\ip}[2]{\left<#1,#2\right>}
\newcommand{\cip}[2]{\left<#1|#2\right>}
\newcommand{\coip}[3]{\left<#1\left|#2\right|#3\right>}

\newcommand{\ket}[1]{\left|#1\right>}
\newcommand{\bra}[1]{\left<#1\right|}
\newcommand{\ketbra}[2]{\left|#1\right>\left<#2\right|}

\newcommand{\valuepermittivity}{\left(8.85419\E{-12}\frac{\C^2}{\J\m}\right)}
\newcommand{\valuehbar}{\left(1.05457\E{-34}\J\s\right)}
\newcommand{\valueplanck}{\left(6.62607\E{-34}\J\s\right)}
\newcommand{\valuefundamentalcharge}{\left(1.60218\E{-19}\C\right)}
\newcommand{\valuemassofelectron}{\left(9.10939\E{-31}\kg\right)}
\newcommand{\valuemassofproton}{\left(1.67262\E{-27}\kg\right)}
\newcommand{\valuespeedoflight}{\left(3\E{8}\vel\right)}
\newcommand{\valueboltzmann}{\left(1.3807\E{-23}\frac{\J}{\K}\right)}
\newcommand{\valueraleigh}{\left(1.097\E{7}\m^{-1}\right)}
\newcommand{\valuegasconstant}{8.31\frac{\J}{\mol\K}}

\newcommand{\operatormomentum}{\left(\frac{\hbar}{i}\drv{}{x}\right)}

\renewcommand{\exp}[1]{\left<#1\right>}

\newcommand{\sech}{\,\text{sech}\,}

\renewcommand{\choose}[2]{\left(\begin{matrix}#1\\#2\end{matrix}\right)}

\newcommand\cancel{\bgroup \markoverwith{---}\ULon}


\newcommand{\mmat}[4]{
\begin{matrix}
#1 & #2\\
#3 & #4\\
\end{matrix}
}

\newcommand{\bmat}[4]{
\begin{bmatrix}
\,\,\,\, #1 \,\, & \,\, #2 \,\,\,\, \\
\,\,\,\, #3 \,\, & \,\, #4 \,\,\,\, \\
\end{bmatrix}
}


\newcommand{\pmat}[4]{
\begin{pmatrix}
\,\,\,\, #1 \,\, & \,\, #2 \,\,\,\, \\
\,\,\,\, #3 \,\, & \,\, #4 \,\,\,\, \\
\end{pmatrix}
}

\newcommand{\bmattt}[9]{
\begin{bmatrix}
\,\,\,\, #1 \,\, & \,\, #2 \,\, & \,\, #3 \,\,\,\, \\
\,\,\,\, #4 \,\, & \,\, #5 \,\, & \,\, #6 \,\,\,\, \\
\,\,\,\, #7 \,\, & \,\, #8 \,\, & \,\, #9 \,\,\,\, \\
\end{bmatrix}
}


\newcommand{\pmattt}[9]{
\begin{pmatrix}
\,\,\,\, #1 \,\, & \,\, #2 \,\, & \,\, #3 \,\,\,\, \\
\,\,\,\, #4 \,\, & \,\, #5 \,\, & \,\, #6 \,\,\,\, \\
\,\,\,\, #7 \,\, & \,\, #8 \,\, & \,\, #9 \,\,\,\, \\
\end{pmatrix}
}

\newcommand{\bvec}[2]{
\begin{bmatrix}
#1 \\
#2 \\
\end{bmatrix}
}
\newcommand{\bveccc}[3]{
\begin{bmatrix}
#1 \\
#2 \\
#3 \\
\end{bmatrix}
}

\newcommand{\bvecccc}[4]{
\begin{bmatrix}
#1 \\
#2 \\
#3 \\
#4 \\
\end{bmatrix}
}

\newcommand{\pvec}[2]{
\begin{pmatrix}
#1 \\
#2 \\
\end{pmatrix}
}
\newcommand{\pveccc}[3]{
\begin{pmatrix}
#1 \\
#2 \\
#3 \\
\end{pmatrix}
}
\newcommand{\pvecccc}[4]{
\begin{pmatrix}
#1 \\
#2 \\
#3 \\
#4 \\
\end{pmatrix}
}


\newcommand{\btvec}[2]{
\begin{bmatrix}
\,\,\,\, #1 \,\, & \,\, #2 \,\,\,\, \\
\end{bmatrix}
}

\newcommand{\paren}[1]{\left(#1\right)}

\newcommand{\pr}[1]{#1^\prime}
\newcommand{\dpr}[1]{#1^{\prime\prime}}

\newcommand{\halfpi}{\frac{\pi}{2}}
\newcommand{\twopi}{2\pi}

\newcommand{\vn}{\vec\nabla}
\newcommand{\vnabla}{\vec\nabla}

\newcommand{\cippsiket}[4]{\left<\psi_{#1}^{(#2)}|\psi_{#3}^{(#4)}\right>}
\newcommand{\coippsiket}[5]{\left<\psi_{#1}^{(#2)}|#3|\psi_{#4}^{(#5)}\right>}
\newcommand{\psiketord}[2]{\ket{\psi_{#1}^{(#2)}}}
\newcommand{\psibraord}[2]{\bra{\psi_{#1}^{(#2)}}}

\newcommand{\psiketordt}[3][t]{\ket{\psi_{#2}^{(#3)}(#1)}}
\newcommand{\psibraordt}[3][t]{\bra{\psi_{#2}^{(#3)}(#1)}}

\newcommand{\coippsikett}[6][t]{\left<\psi_{#2}^{(#3)}(#1)|#4|\psi_{#5}^{(#6)}(#1)\right>}

\newcommand{\paslash}{\ensuremath \raisebox{0.025cm}{\slash}\hspace{-0.25cm}\partial\/}

\newcommand{\sll}[1]{\rlap{\hbox{$\mskip 1 mu /$}}#1}      % good slash for lower case
\newcommand{\Sl}[1]{\rlap{\hbox{$\mskip 3 mu /$}}#1}      % " upper
\newcommand{\SL}[1]{\rlap{\hbox{$\mskip 4.5 mu /$}}#1}    % " fat stuff (e.g., M)



%%% Local Variables: 
%%% mode: latex
%%% TeX-master: t
%%% End: 
%@-node:@file /Users/cog/Documents/macros.tex
%@-leo

%@-node:gcross.20091029172439.1256:<< Macros >>
%@nl

\begin{document}

%@+others
%@+node:gcross.20090513124712.1:Title Page
\title{CodeQuest}

\author{Gregory M. Crosswhite}
\affiliation{Department of Physics\\ University of Washington\\ Seattle, 98185}

\author{Dave Bacon}
\affiliation{Department of Computer Science \& Engineering \\ Department of Physics \\ University of Washington \\ Seattle, 98185}

%\pacs{03.67.-a}

\email{gcross@phys.washington.edu, dabacon@cs.washington.edu}


\maketitle

\newpage

\tableofcontents
%@-node:gcross.20090513124712.1:Title Page
%@+node:gcross.20090405101642.5:Introduction
In the field of quantum computing, there is a grand battle between the
forces of humankind, which seek to reliably store and manipulate
quantum information, and the forces of nature, which generally seek to
destroy it.  Although armies of experimentalists have made hereoic
efforts to build systems that shield quantum information from harm,
nature inevitably manages to get past these defences from time to time
and strike a blow.  This might seem to paint a grim outlook for the
possibility of building a quantum computer, but happily it turns out
to be the case that one can generally repair damage to quantum
information as long as one knows the exact form that the damage took,
and furthermore that one can build a `trap' --- that is to say, a
\emph{quantum code} --- that tricks nature into giving this
information up.

Now, the nature of codes is that they decouple the space in which our
computation lives from the space in which the physical information is
stored; that is to say, although we design our quantum circuit to
operate on some space of qubits $\mathscr{C}$, each of these qubits
does \emph{not} directly correspond to a physical qubit, but rather
there is some isomorphism that relates the entire space $\mathscr{C}$
to the space of physical qubits, $\mathscr{P}$.  To distinguish
between these two spaces, we shall call the space of qubits in whose
terms the computation is expressed the \emph{computational space}, and
the space of qubits which have physically been built the
\emph{physical space}.

Of course, merely building an isomorphism between these two spaces is
not enough to allow us to correct errors.  For one thing, we need to
add extra qubits to the computational space that contain a record of
the damage that we can read out; thus, we shall say that the full
computational space is $\mathscr{C}:=\mathscr{R}\times\mathscr{Q}$,
where the qubits that live in $\mathscr{R}$ have the role of keeping a
record of the errors that have been introduces, and the qubits that
live in $\mathscr{Q}$ are the qubits in whose terms our quantum
algorithm is expressed.  Since we are only performing measurements on
$\mathscr{R}$, we can effectively ignore all operators except, say,
the $Z$ measurement operator for each qubit; this set of commuting
operators allows us to completely measure the state of qubits in
$\mathscr{R}$.  In order to build the `trap' element into our system,
we need to ensure that whenever nature strikes at the physical space
$\mathscr{P}$, it is isomorphic to a strike on the computational space
that leaves a \emph{measureable} record in $\mathscr{R}$, which means
in particular that it is isomorphic to an operator that must
\emph{anti-commute} with the $Z$ operator (or whatever else we have
chosen to be our basis of measurement) of one of the qubits in
$\mathscr{R}$.  Note that although we speak of measuring the qubits in
$\mathscr{R}$, they of course cannot be measured directly, but instead
we take the measurement operator of interest in $\mathscr{R}$ and
measure the \emph{isomorphic} operator in the physical space
$\mathscr{P}$; this isomorphic operator is referred to as a
\emph{stabilizer}, and the full set of operators on $\mathscr{P}$
which are isomorphic to our chosen measurement operators on
$\mathscr{R}$ are referred to as the \emph{stabilizers} of the code.

Up to this point, the formalism we have described is known as
\emph{stabilizer codes} and its essential characteristic is the
forcing of every qubits in $\mathscr{R}$ to always have a definite
value in some basis by performing continuous measurement.  What if,
however, we relaxed this constraint and only continuously measured
some of the qubits in $\mathscr{R}$?  That is to say, what if we split
the qubits in $\mathscr{R}$ into two catagories: \emph{stabilizer
qubits} whose states we care about and force to always have a definite
value in some basis through continuous measurement, \emph{gauge
qubits} whose states we do not care about.  (The latter get their name
from the fact that they provide a `gauge' degree of freedom, i.e. a
degree of freedom that is irrelevent to us.)  Then we would have that
$\mathscr{R}=\mathscr{S}\times \mathscr{G}$, where $\mathscr{S}$ is
the space in which the stabilizer qubits live, and $\mathscr{G}$ is
the space in which the so-called gauge qubits live; such a scheme is
known as a \emph{subsystem code}.  In this case, we shall use the term
\emph{stabilizers} to denote the set of operators in $\mathscr{P}$
which are isomorphic to our chosen measurement operators of interest
in $\mathscr{S}$.

At first there might not seem to be an advantage to this approach,
since it essentially means adding qubits to our code that are
`wasted'; however, in practice this can actually make our code easier
to implement in a physical system.  The reason for this is that often
the measurement of stabilizers requires performing operations that
involve several qubits at once, which can be difficult or impossible
to implement\footnote{See, for example, the \emph{toric code} [ref],
which uses 4-qubit measurements.}  However, there are ways that by
adding additional qubits, one can instead use a set of, say, 2-qubit
operators whose simultaneous measurement results in an effective
measurement of all of the stabilizers so that the stabilizer qubits
are all collapsed to definite values in our chosen basis\footnote{For
examples of this, see the compass model code [ref].}.

What makes this approach powerful is that we no longer need our
measurements on the physical system to commute with each other, as
long as they all commute with the stabilizers, since then the fact
that they do not commute only affects the gauge qubits, which we do
not care about.  In fact, it is so powerful that any set of
measurements that are members of the Pauli group on the physical
qubits can be used to implement a subsystem code, and in fact we can
compute the code that it implements, as we shall prove in this paper;
of course, the resulting code might not be useful --- since among
other possibilities, it might be that it has no room for encoding the
quantum information that we want to store --- but it definitely
exists.  This fact invites an approach to finding useful subsystem
codes that is in many ways opposite to the approach commonly taken:
rather than coming up with codes and then trying to figure out how
they might be physically implemented, why not start with a class of
physical implementations and search within it for useful subsystem
codes?  This is the approach that we explore in this paper.

In the first section, we shall formally prove that every set of
measurements that are members of the Pauli group acting on the system
give rise to a subsystem code, and we shall in the process present
algorithms for computing this code (or at least, for computing one
such code, since it is not unique) and its distance.  In the second
section, we shall present numerical results obtained by applying a
code implementing this algorithm to explore systems built using
lattices that take the form of te 11 regular tilings.  In the third
section, we shall present an algorithm for seaching over all of the
systems that can be implemented by using arbtitrary Ising interactons
with the structure of a graph, and then we shall present the results
that we have obtained from our searches.
%@-node:gcross.20090405101642.5:Introduction
%@+node:gcross.20090423002455.2:Algorithm
\section{Algorithm}

%@+node:gcross.20090423002455.3:Stabilizers and gauge qubits
\subsection{Construction of the subsystem code}

\begin{remark}
This subsection describes by way of a constructive proof how one can compute the quantum subsystem code implemetable by a the set of measurement operators.  For a listing of pseudo-code that implements the algorithm described in this proof, see Table \ref{code-algorithm} near the end of this subsection.
\end{remark}
%@+node:gcross.20090520163423.14:Introduce main theorem
Although conceptually a subsystem code is an isomorphism $T$ such that  $\mathscr{P}\approx^T \mathscr{S}\times\mathscr{G}\times\mathscr{Q}$ --- that is, an isomorphism between the \emph{physical} space of qubits and the \emph{computational} space of qubits in whose terms our computation is actually expressed, we do not need to actually construct this isomorphism in order to be able to use the code.  Since all of our work will be done on the physical system anyway, it suffices to know the operators in the physical space $\mathscr{P}$ that are isomorphic to the qubit measurement operators of interest in the computational space $\mathscr{S}\times\mathscr{G}\times\mathscr{Q}$, and it is exactly the operators on $\mathscr{P}$ that the algorithm we present shall compute\footnote{If one really wanted to, one could explicitly construct the isomorphism $\mathscr{T}$ from these operators by computing the unitary operator which simultaneously diagonalizes a the maximal subset of commuting measurements from this set of operators on $\mathscr{P}$, but in practice this is not particularly useful.}.

When one wants to define a qubit in terms of its measurement operators, it suffices to define two operators that anti-commute with each other but which commute with all of the others measurement operators that have been defined, since this gives us the $X$ and $Z$ measurements on the qubit which are sufficient to generate the full $Pauli$ group (minus phases).  Since working with such pairs of operators shall be a common theme in our algorithm, we shall introduce the following definition in order to simplify the language used to describe them.

\begin{definition} A pair of operators is a \emph{conjugal pair in relation to the set} $\set X$ when each of the operators in the pair commutes with every operator in $\set X$ except for its \emph{conjugal partner} --- that is, the other operator in the conjugal pair --- should its conjugal partner be a member of $\set X$.
\label{conjugal-pair-definition}
\end{definition}

Note that we have explicitly not required that the operators in the conjugal pair be members of $\set X$ in order to be a conjugal pair in relation to it.  However, should both operators be members of $\set X$, then neither operator can belong to a different conjugal pair with respect to $\set X$, since in that case there would be an operator in $\set X$ (namely, its original conjugal partner) with which it anti-commutes that was not its conjugal partner in the new pair, leading to a contradiction.

For convenience, we introduce the following additional definitions:

\begin{definition}

\begin{enumerate}
\item $\pauligroup$ is the group of Pauli operators --- that is, the group of tensor products of the (unnormalized) Pauli matrices --- acting on the physical space $\mathscr{P}$;
\item $\powerset(\set S)$ is the power set of $\set S$, i.e. the set of all subsets of $\set S$; and
\item $\centralizer_\mathfrak{G}(\set S)$ is the centralizer of $\set S$, that is the subgroup of elements in $\mathfrak{G}$ which commute with $\set S$;
\item the function $\genfun:\powerset(\pauligroup)\to\powerset(\pauligroup)$ is defined such that $\genfun(\set S)$ is the set of all possible products of operators in $\set S$ --- that is, it is the set \emph{generated} by $\set S$.
\end{enumerate}

\end{definition}

We now introduce the main theorem of this subsection.

\begin{theorem} \label{theorem-SG} Suppose we are given a sequence of Pauli operators, $\lst O$.  Then there exist sets of Pauli operators $\set S\subseteq\pauligroup$, $\set G\subseteq\pauligroup$, and $\set L\subseteq\pauligroup$ such that
\begin{enumerate}
\item each of the operators in $\set S \cup \set G \cup \set L$ is independent from the rest --- i.e., no operator in this (unioned) set can be written as a product of other operators in the set;
\item each operator in $\set L \cup \set G$ is a member of a conjugal pair in relation to $\set S \cup \set G \cup \set L$;
\item $\genfun(\set S \cup \set G)=\genfun\paren{\{\lst O_i\}}$\footnote{Here we use the notation $\{\vec{O}_i\}$ to refer to the set of elements in the sequence $\vec{O}$.};
\item and $\genfun(\set S \cup \set G \cup \set L)=\centralizer_\pauligroup(\set S)$
\end{enumerate}
\end{theorem}

\begin{remark}
The main work in the proof of this theorem will be performed by proving several related propositions.  First we shall show how the set $\set G$ and a sequence $\lst S$ are constructed from the sequence of operators $\lst O$.  Since we want our stabilizers to form an independent set of operators, we shall then show that through a Gaussian elimination procedure it is possible to extract a list of independent operators from a sequence $\lst S$ resulting in a set $\set S$.  Finally, we shall show how using this same Gaussian elimination procedure we can transform a subset of the operators of $\set S\cup\set G$ into a form that makes it trivial to compute the logical qubit operators $\set L$.
\end{remark}
%@nonl
%@-node:gcross.20090520163423.14:Introduce main theorem
%@+node:gcross.20090519160701.3:Construction of sequences
\begin{proposition} \label{proposition-SG} Suppose that we are given a sequence of Pauli operators $\lst O\subseteq \pauligroup$.  Then there exists a sequence of Pauli operators $\lst S\subseteq\pauligroup$ and a set of Pauli operators $\set G\subseteq\pauligroup$ such that
\begin{enumerate}
\item all of the operators in $\lst S$ commute with each other and also all of the operators in $\lst G$; \label{stabs-commute-with-G}
\item each operator in $\set G$ is a member of a \emph{conjugal pair} (Definition \ref{conjugal-pair-definition}) in relation to $\{\lst S_i\} \cup \set G $ \label{conjugal-pairs-commute-with-SAG}; and
\item $\genfun\paren{\{\lst S_i\}\cup \set G}=\genfun\paren{\{\lst O_i\}}$ \label{SAG-spans-all}.
\end{enumerate}
\end{proposition}

\begin{proof}
Proof by induction.  For the base case, note that if $\lst O$ is empty then $\lst S:=\emptyset$ and $\set G:=\emptyset$ trivially satisfy all properties.

Now assume that the proposition holds for a sequence of length $n-1$, and consider a sequence of operators $\lst O$ of length $n$.  By the inductive hypothesis, we know that there is a sequence $\lst S'$ and a set $\set G'$ satisfying the properties above for the subsequence of $\lst O$ consisting of the first $n-1$ operators.  Let $o:=\lst O_n\cdot \prod_{g\in \set G, \{\lst O_n,g\}=0} \text{conj}_{\set G}(g)$ --- that is, the product of $\lst O_n$ with the conjugal partner of every operator in $\set G$ with which $\lst O_n$ anti-commutes.  This definition guarantees that $o$ commmutes with every operator in $\set G$;  furthermore, we can obtain $\lst O_n$ back from $o$ since every operator in $\set G$ squares to the identity and thus $\lst O_n=o\cdot \prod_{g\in \set G, \{\lst O_n,g\}=0} \text{conj}_{\set G}(o)$; therefore we conclude that $\genfun\paren{\{\lst S'_i\} \cup \set G' \cup \{o\}}=\genfun\paren{\{\lst O_i\}}$.

If $o$ commutes with every operator in $\lst S'$, then set
$$\lst S_i :=
\begin{cases}
\lst S'_i & i \le n-1 \\
o & i = n
\end{cases}
$$
and $\set G := \set G'$, and we are done.  Otherwise, let $s$ be some operator in $\lst S'$ that anti-commutes with $o$, $\set G:=\set G'\cup \{s,o\}$
\footnote{Observe that neither $o$ nor $s$ can be present in $\set G'$ since they commute with every operator in $\set G'$, so the new set $\set G:=\set G'\cup \{s,o\}$ gives us a strictly larger set.  This fact is irrelevant far as the proof is concerned, but it has the important consequence that a computer code implementing the algorithm described by this proof can append $s$ and $o$ to a list of gauge operators and assume that this list continues to form a set (i.e., a sequence without duplicates) without having to explicitly check for this.}, $\lst S_i'' := f(\lst S'_i)$, and $\lst S$ be the subsequence of $\lst S''$ with the identity operators removed, where
$$
f(s') :=
\begin{cases}
s'\cdot s & \{s',o\}=0\\
s' & \text{otherwise}
\end{cases}.
$$
Observe that by this definition, all of the operators in $\lst S$ commute with every operator in $\set G$, so property \ref{stabs-commute-with-G} is satisfied.  Since the only difference between $\set G'$ and $\set G$ is the addition of $s$ and $o$, which form a conjugal pair with respect to $\{\lst S_i\} \cup \set G$, we conclude that property \ref{conjugal-pairs-commute-with-SAG} is satisfied.
Lastly, since $s\in \set G$, we can form any operator in $\lst S'$ with products of operators in $\lst S$ and $\set G$, so therefore $\genfun\paren{\{\lst S_i\} \cup \set G}=\genfun\paren{\{\lst S'_i\} \cup G' \cup \{s,o\}}=\genfun\paren{\{\lst O_i\}}$, and so the final property is satisfied.

We conclude by noting that since all of the operators in $\lst S$ and $\set G$ were formed from products of operators in $\lst O$, which are Pauli operators (i.e., members of the group $\pauligroup$), they are Pauli operators themselves.
\end{proof}
%@-node:gcross.20090519160701.3:Construction of sequences
%@+node:gcross.20090519160701.4:Making them independent
\begin{remark}
A consequence of not requiring independence of the operators in $\lst O$ is that the operators $\lst S$ given by Proposition \ref{proposition-SG} are not necessarily independent.  Happily, since all of these operators can be expressed as tensor products of Pauli operators, we can construct a set of independent operators by performing an analog of Gaussian elimination.
\end{remark}

\begin{proposition}
\label{make-independent-using-elimination}
Suppose that we have been given a sequence of Pauli operators which commute with each other, $\lst R$.  Then there exists
\begin{enumerate}
\item a sequence $\lst S$ of $n$ independent operators such that $\genfun\paren{\{\lst S_i\}}=\genfun\paren{\{\lst R_i\}}$,
\item a sequence of $n$ integers without duplicates in the inclusive range $1\dots n$,
\item and a map $p:\{1\dots n\} \to \{0,1\}$ such that $\lst S_i$ is the only operator in $\lst S$ that anti-commutes with $P_{k_i}^{[p(i)]}$, where $P_k^{[0]}:=X_k$ and $P_k^{[1]}:=Z_k$.
\end{enumerate}
\end{proposition}

\begin{proof}
Proof by induction.  For the base case, we observe that if $\lst R$ is empty, then the trivial sequences $\lst S:=\emptyset$ and $\lst k :=\emptyset$ and the trivial function $p:\emptyset\to\emptyset$ satisfy the requirements.

Now suppose that we know the proposition holds for sequences of length $N-1$, and we are given a sequence $\lst S$ of length $N$.  By our inductive hypothesis, we can apply the proposition to the first $N-1$ operators in $\lst R$ obtain sequences $\lst S'$ and $\lst k'$ of length $n-1$\footnote{$n\ne N$ in general}, and a map $p':\{1\dots n-1\}\to \{0,1\}$ which all satisfy the respective properties of the theorem.  Let $$s:=\lst R_N\cdot \prod_{i=1\dots n-1, \,\,\left\{\lst R_N,P_{k'_i}^{[p(i)]}\right\}=0} \lst S'_i.$$  We know that $s$ commutes with every operator in $\lst S'$ because both $s$ and every operator in $\lst S'$ are equal to products of operators in $\lst R$, which all commute with each other.  Furthermore, since $s$ is a product of $\lst R_N$ and a factor of $\lst S'_i$ for every $i$ such that $\lst R_N$ and $P_{k'_i}^{[p'(i)]}$ anti-commute, and we know that $\lst S_i'$ is the only operator in $\lst S'$ that anti-commutes with $P_{k'_i}^{[p'(i)]}$ for $i=1\dots n-1$, it is therefore the case that $s$ commutes with every member of the set $\{P_{k'_i}^{[p'(i)]}\}_{i=1\dots n-1}$.  Finally, since $s$ is a product of $\lst R_N$ and operators in $\lst S'$, we can obtain $\lst R_N$ entirely from products of operators in $\{\lst S'_i\} \cup \{s\}$, and so $\genfun\paren{\{\lst S'_i\} \cup \{s\}}=\genfun\paren{\{\lst R_i\}}$.

If $s$ is the identity operator, then let $\lst S:=\lst S'$ and $p:=p$ and we are done.  Otherwise, we shall now show that there must exist integers $j\in\{1,\dots,N\}\backslash\{\lst k'_i\}$ and $l\in\{0,1\}$ such that $s$ anti-commutes with $P_{j}^{[l]}$, by demonstrating that if this were not the case then $s$ would have to anti-commute with some element in $\lst S'$, leading to a contradiction.

Assume that $s$ commutes with every operator in the set $\left\{P_j^{[l]}:\quad j\in\{1,\dots,N\}\backslash\{\lst k'_i\}, \quad l\in\{0,1\}\right\}.$  Recalling that $s$ is a member of the Pauli group and thus a tensor product of single-particle Pauli spin matrices, and also that $s$ commutes with every member of the set $\{P_{\lst k'_i}^{[p'(i)]}\}_{i=1\dots n-1}$, we see therefore that $s$ must be a product of elements from this set --- that is, there is some subset $\emptyset \ne \set F \subseteq \{P_{\lst k'_i}^{[p'(i)]}\}_{i=1\dots n-1}$ such that $s=\prod_{o\in \set F} o$.  However, from our inductive hypothesis we know that for every operator $f\in\set F$ there is an operator $s'\in\lst S'$ that anti-commutes with $f$ but commutes with the operators in $\set F\backslash\{f\}$.  Since $s$ is therefore a product of a single operator that anti-commutes with $s'$ and more operators that commute with $s'$, we conclude that $s$ and $s'$ anti-commute, which contradicts our earlier conclusion that $s$ commutes with every operator in $\lst S'$.

Now that we have shown that there exist integers $j\in\{1,\dots,N\}\backslash\{\lst k'_i\}$ and $l\in\{0,1\}$ such that $s$ anti-commutes with $P_{j}^{[l]}$, in terms of these integers we define
$$
\begin{aligned}
\lst S_i &:= 
\begin{cases}
\begin{cases}
\lst S'_i \cdot s & \{\lst S_i',P_j^{[l]}\}=0 \\
\lst S'_i & \text{otherwise}
\end{cases} & 1\le i\le n-1 \\
S' & i=n
\end{cases}, \\
\lst k_i &:=
\begin{cases}
\lst k'_i & 1 \le i \le n-1 \\
j & i=n
\end{cases},\quad \text{and} \\
p(i) &:=
\begin{cases}
p'(i) & 1 \le i \le n-1\\
l & i=n
\end{cases},
\end{aligned}
$$ and we are done.
\end{proof}
%@-node:gcross.20090519160701.4:Making them independent
%@+node:gcross.20090527164539.1:Construct the logical operators
\begin{remark}
Proposition \ref{make-independent-using-elimination} is good for more than computing an independent set of generators from a commuting list of operators;  it is also the key ingrediant in computing the logical qubit operators.
\end{remark}

\begin{proposition}
\label{construction-of-logicals}
Suppose that we have been given the objects described in 1-3 of Proposition \ref{make-independent-using-elimination}.  Let $\set S := \{\vec S_i\}_i.$  Then there exists a set of operators $\set L$ such that
\begin{enumerate}
\item \label{L-are-independent} the operators in $\set S\cup\set L$ are independent;
\item \label{L-are-conjugal-pairs} every operator in $\set L$ is a member of a conjugal pair with respect to $\set S\cup\set L$;
\item \label{L-completes-the-generators} $\genfun\paren{\set S\cup\set L}=\centralizer_\pauligroup(\set S)$ --- that is, the set generated by $\set S\cup\set L$ is equal to the set of Pauli operators that commute with $\set S$.
\end{enumerate}
\end{proposition}

\begin{proof}
Recalling that $n$ is the number of elements in $\lst S$ (and $\set S$), let $\lst l$ be some sequential ordering of $\{1 \dots N\}\backslash\{\vec k_i\}_i$, and then let $\set L:=\{\lst A_i\}_i\cup\{\lst B_i\}_i$ where
$$
\begin{aligned}
\lst A_i &:= P_{\lst l_i}^{[1]}\cdot \prod_{j=1\dots n\atop \{P_{l_i}^{[1]},\lst S_j\}=0} P_{\lst k_j}^{[s(j)]},\\
\lst B_i &:= P_{\lst l_i}^{[0]}\cdot \prod_{j=1\dots n\atop \{P_{l_i}^{[0]},\lst S_j\}=0} P_{\lst k_j}^{[s(j)]}.\\
\end{aligned}
$$

To see that property \ref{L-are-independent} is satisfied, observe the following.  First, the operators in $\set L$ are independent from the operators in $\set S$ since none of them is the identity operator and they all commute with every operator in $\{P_{\lst k_i}^{[s(i)]}\}_{i=1 \dots n}$.  Second, they are independent from each other since for every $i=1 \dots |\lst l|$ we have that $\vec A_i$ is the only operator that anti-commutes with $P_{\lst l_i}^{[0]}$ and $\lst B_i$ is the only operator that anti-commutes with $P_{\lst l_i}^{[1]}$.  Thus we conclude that all of the operators in $\set S\cup\set L$ are independent.

Next, to see that property \ref{L-are-conjugal-pairs} holds, observe that for every choice of operators $\lst A_i$ and $\lst S_j$ we have (by intentional construction) that $\lst S_j$ either anti-commutes with two of the operators in the product forming $\lst A_i$ or none at all, and so $[\lst S_i,\lst A_j]=0$ for all $i=1\dots n$ and $j=1\dots |\lst l|$;  by the same reasoning we see also that $[\lst S_i,\lst B_j]=0$ for all $i=1\dots n$ and $j=1\dots |\lst l|$.  Furthermore, each operator $\lst A_i$ commutes with every operator in $\set L$ except for its conjugal partner $\lst B_i$, since the only factor in $\lst A_i$ that could anti-commute with a factor contained within another operator in $\set L$ is $P_{l_i}^{[1]}$, and $\lst B_i$ is the only operator in $\set L$ that contains a factor $P_{l_i}^{[0]}$ that anti-commutes with $\lst X_{l_i}$;  reversing this argument, we also see that $\lst B_i$ commutes with every operator in $\set L$ except for $\lst A_i$.  Thus, every operator in $\set L$ is a member of a conjugal pair with respect to $\set L\cup\set S$.

Finally, to see that property \ref{L-completes-the-generators} holds, observe that since the operators in $\set S$ commute they can therefore be simultaneously diagonalized, which means that there is an automorphism on $\pauligroup$ that takes $\lst S_i\mapsto P_i^{[1]}$ for every $i=1 \dots n$.  The only operators that commute with every such $P_i^{[1]}$ are those which do not contain any factor of $P_i^{[0]}$ for $i=1 \dots n$, and so $\centralizer_\pauligroup\paren{\{P_i^{[0]}\}_{i=1\dots n}} = \genfun\paren{\{P_i^{[1]}\}_{i=1 \dots n}\cup \{P_i^{[l]}\}_{i=n+1 \dots N, \,\, l=0,1}}$, which has $2N-n$ generators.  Since the automorphism preserves the number of generators in the centralizer, we thus conclude that $\centralizer_\pauligroup(\set S)$ has exactly $2N-n$ generators.  Since $\set S\cup\set L$ contains independent operators which commute with every member of $\set S$, and furthermore $|\set S\cup\set L|=2N-n$, we thus conclude that $\genfun\paren{\set S\cup\set L}=\centralizer_\pauligroup(\set S)$.
\end{proof}
%@nonl
%@-node:gcross.20090527164539.1:Construct the logical operators
%@+node:gcross.20090520163423.15:Prove main theorem
With these building blocks in place, we now prove the main theorem:

\begin{proof}[Proof of Theorem \ref{theorem-SG}]
By Proposition \ref{proposition-SG}, we know that there exists a list of operators $\lst S$ and a set of independent operators $\set G$ satisfying the properties that are listed there.  By Proposition \ref{make-independent-using-elimination}, we know that there is an independent set of operators $\set S$ that generate the same subgroup as $\lst S$.  

Now let $\set F$ be a maximal subset of commuting operators in $\set G$ --- i.e., for each conjugal pair in $\set G$ take one of the two operators --- and then let $\set O := \set F \cup \set S$.  Since all of the operators in $\set O$ commute, we apply Proposition \ref{make-independent-using-elimination} again to conclude the existance of the objects listed there, and then we immediately apply Proposition \ref{construction-of-logicals} to show that a set $\set M$ exists with the properties listed there.  We are not done yet, however, since there might be operators in $\set G$ with which operators in $\set M$ anti-commute, so we let
$$\set L := \{m\cdot\prod_{f\in \set F\atop \{M,\text{conj}_{\set G}(f)\}=0} f :\quad m \in\set M\}$$
where $\text{conj}_{\set G}(F)$ is the conjugal partner of $F$ in the set $\set G$.  This guarantees that the operators in $\set L$ commute with every operator in $\set S\cup\set G$, and so we are done.
\end{proof}
%@-node:gcross.20090520163423.15:Prove main theorem
%@+node:gcross.20090511123440.5:Pseudo-code
\begin{remark}
A pseudo-code representation of the algorithm described by Theorem \ref{theorem-SG} is given in Table \ref{SG-algorithm}.
\end{remark}

\begin{table}
\label{SG-algorithm}
\begin{codebox}
\Procname{$\proc{Compute-Subsystem-Code}(\lst O)$}
\li $\lst S \gets []$
\li $\lst G\gets []$
\li \For $o \gets \lst O$ \label{li:csg-main-loop-start}
\li \Do
\li      \For $(g_X, g_Z) \gets \lst G$ %\Comment i.e., iterate over conjugal pairs in $\lst G$
\li      \Do
\li        \kw{if}  $\func{anti}(o,g_X)$ \kw{then} $o \gets o \cdot g_Z$
\li        \kw{if} $\func{anti}(o,g_Z)$ \kw{then} $o \gets o \cdot g_X$
          \End 
\li      \kw{if} \text{$o$ is identity} \kw{then} \Goto \ref{li:csg-main-loop-start}
\li      \For $s \gets \lst S$
\li      \Do
\li        \kw{if} $\func{anti}(o,s)$ \kw{then} \Goto \ref{li:make-into-gauge}
          \End
\li      \Goto \ref{li:csg-main-loop-start}
\li      $\lst G \gets \lst G \cup [(o,s)]$ \label{li:make-into-gauge}
\li      $i \gets 1$ 
\li      \For $s' \gets \lst S$ \label{li:kill-redundant-stabs}
\li      \Do
\li            \kw{if} $s'=s$ \kw{then} \Goto \ref{li:kill-redundant-stabs}
\li            \If $\func{anti}(s',o)$
\li            \Then
\li                  $\lst S[i] \gets s' \cdot s$
\li            \Else
\li                  $\lst S[i] \gets s$
                 \End
\li            $i \gets i+1$
             \End
\li        delete $\lst S[i\dots|\lst S|]$
      \End
\li $\lst I \gets []$
\li $\lst P \gets []$
\li \kw{call} $\proc{Gaussian-Elimination}(\lst S,1,\lst I,\lst P)$
\li $\lst T \gets \lst S \cup [g_X | (g_X,g_Z) \in \lst G]$
\li \kw{call} $\proc{Gaussian-Elimination}(\lst T,|\lst S|+1,\lst I,\lst P)$
\li $\lst L \gets []$
\li \For $i\gets 1\dots\,\,\text{number of physical qubits}$ \label{li:compute-logicals-loop}
\li \Do
\li     \kw{if} $i\in\lst I$ \kw{then} \Goto \ref{li:compute-logicals-loop}
\li     $l_X \gets X_i$
\li     $l_Z \gets Z_i$
\li     \For $(j,p,t)\gets (\lst I,\lst P,\lst T)$
\li     \Do
\li        \If $p=0$
\li        \Then
\li            \kw{if} $\func{anti}(t,X_j)$ \kw{then} $l_X \gets l_X \cdot Z_j$
\li            \kw{if} $\func{anti}(t,Z_j)$ \kw{then} $l_Z \gets l_Z \cdot Z_j$
\li        \Else
\li            \kw{if} $\func{anti}(t,X_j)$ \kw{then} $l_X \gets l_X \cdot X_j$
\li            \kw{if} $\func{anti}(t,Z_j)$ \kw{then} $l_Z \gets l_Z \cdot X_j$
             \End
          \End 
\li     \For $(g_X,g_Z)\in\lst G$
\li     \Do
\li         \kw{if} $\func{anti}(l_X,g_Z)$ \kw{then} $l_X \gets l_X \cdot g_X$
\li         \kw{if} $\func{anti}(l_Z,g_Z)$ \kw{then} $l_Z \gets l_Z \cdot g_X$
          \End
      \End
\li \Return $(\lst S,\lst G,\lst L)$
\end{codebox}
\caption{Algorithm which computes the subsystem code generated by a given list of measurement operators $\lst O$.}
\end{table}
%@-node:gcross.20090511123440.5:Pseudo-code
%@-node:gcross.20090423002455.3:Stabilizers and gauge qubits
%@+node:gcross.20090511123440.3:Optimal generators
\subsection{Optimal generators}

\label{optimal-generators}

%@+others
%@+node:gcross.20091221145013.1274:Lemma:  Recombining generators
In general there are multiple sets of operators that satisfy the properties of \ref{theorem-SG}, as is illustrated by the following Lemma:

\begin{lemma}
\label{combining-pairs}
Given conjugal pairs $Q:=(a,b)$ and $R:=(c,d)$ in relation to some set $\set X$ such that either $a\ne c$ or $b\ne d$, we have that
\begin{enumerate}
\item the pairs $Q':=(a\cdot c,b)$ and $R':=(c,d\cdot b)$ are conjugal pairs with respect to $\set X \backslash \{Q,R\} \cup \{Q',R'\}$; and
\item $\genfun\paren{\{a,b,c,d\}}=\genfun\paren{\{a\cdot c,b,c,d\cdot b\}}$.
\end{enumerate}
\end{lemma}

\begin{proof}
\begin{enumerate}
\item Since $[a,c]=[a,d]=[b,c]=[b,d]=\{a,b\}=\{c,d\}=0$, we see therefore that $[a\cdot c,c]=[a\cdot c,d\cdot b]=[b,c]=[b,d\cdot b]=\{a\cdot c,b\}=\{c,d\cdot b\}=0$.  Furthermore, since $a$, $b$, $c$ and $d$ commute with every operator in $\set X\backslash \{Q,R\}$, so do $a\cdot c$ and $d\cdot b$.
\item Since $b$ and $c$ are Pauli operators and thus square to the identity, we have that $a\cdot c\cdot c=a$ and $d\cdot b\cdot b=d$, and so $\genfun\paren{\{a,b,c,d\}}=\genfun\paren{\{a\cdot c,b,c,d\cdot b\}}$.
\end{enumerate}
\end{proof}
%@-node:gcross.20091221145013.1274:Lemma:  Recombining generators
%@+node:gcross.20091221145013.1275:Definition: Undetectable error
As a result of this lemma, we see that we can take pairs of arbitrary conjugal pairs from sets $\set G$ and $\set L$ of Theorem \ref{theorem-SG} and replace them with different pairs per the recipe in Lemma \ref{combining-pairs} such that the properties of the theorem still hold.  So given that these sets are not unique, the natural question is:  What is the best choice of $\set G$ and $\set L$?  To answer this, we observe that another criteria we would like for our code to satisfy is that it be as robust to errors as possible;  in particular, we seek to maximize the difficulty of \emph{undetectable errors}, which is defined as follows:

\begin{definition}
Given a set $\set S\subseteq\pauligroup$ and operators $l\in\pauligroup$ and $e\in\centralizer_\pauligroup(\set S)$ which anti-commute (i.e., $\{l,e\}=0$), we say that $e$ is an \emph{undetectable error with respect to} $\set S$ \emph{acting on} $l$.
\end{definition}
%@nonl
%@-node:gcross.20091221145013.1275:Definition: Undetectable error
%@+node:gcross.20091221145013.1276:Definition: Weight
We assume that the `difficulty' of an interaction between our physical system and its environment is related to the number of physical qubits in our system that are participating in the interaction.  Thus, the natural metric for measuring the relative difficulty of an error is given by its weight, which recall is defined as follows:

\begin{definition}
Given an operator $p\in\pauligroup$---which recalls means that $p$ is the tensor product of single-qubit Pauli unnormalized spin matrices---the \emph{weight} of $p$ is the number of single-qubit operators in the product which are non-trivial (i.e., not the identity).  So for example, the weight of $I\otimes I\otimes I$ is 0, the weight of $I\otimes Z\otimes I\otimes X$ is 2, and the weight of $Z\otimes X\otimes Y$ is 3.
\end{definition}
%@nonl
%@-node:gcross.20091221145013.1276:Definition: Weight
%@+node:gcross.20091221145013.1277:Definition: Additional notation
For convenience, we introduce the following additional notation:

\begin{definition}
$\quad$

\begin{itemize}
\item assuming we have a set of independent operators, $\set Q$, the function $\set G_{\set Q}:\genfun(\set Q)\to\powerset(\set Q)$ is defined (uniquely) such that for every $q\in\set Q$ we have that $q=\prod_{o\in\set G_{\set Q}(q)} o$;
\item the function $w:\pauligroup\to \mathscr{N}$ is defined such that $w(o)$ gives the weight of $o$;
\item the function $e_{\set S}:\centralizer_\pauligroup\paren{\set S}\to \paren{\powerset\circ\centralizer_\pauligroup}\paren{\set S}$ is defined such that $\set e_{\set S}(l)$ is the set of minimizers of $w$ over the set $\left\{o: o\in \centralizer_\pauligroup\paren{\set S}, \{o,l\}=0\right\}$ --- that is, it gives the undetectable errors with respect to $\set S$ acting on $l$ that are of minimum weight;
\item the function $\om_{\set S}:\centralizer_\pauligroup\paren{\set S}\to\mathscr{N}$ is defined such that $\om_{\set S}:=w(o)$ for an arbitrarily chosen $o\in\circ \set e_{\set S}$ --- note that function is well-defiend since all operators in the set $\set e_{\set S}$ have the same weight;
\item the function $m_{\set S}:\centralizer_\pauligroup\paren{\set S}\times \centralizer_\pauligroup\paren{\set S} \to \mathscr{N}$ is defined such that $m_{\set S}(l,l'):=\min \{\om_{\set S}(l),\om_{\set S}(l')\}$ --- that is, it gives the smaller of the weights of the smallest weight errors acting on respectively $\set L$ and $\set L'$;
\item the function $\vec M^{(N)}_{\set S}$ is defined such that $\vec M^{(N)}_{\set S}\paren{\vec P}$ is the sequence of $N$ integers such that $\vec M^{(N)}_{\set S}\paren{\vec P}_i := m_{\set S}\paren{\vec P_i}$;
\item the functions $p_1$ and $p_2$ are defined such that, given $(a,b):=x$, we have that $p_1(x):=a$ and $p_2(x):=b$.
\item the function $\set U:\powerset(\pauligroup\times\pauligroup)\to\powerset(\pauligroup)$ is defined (for convenience) such that $\set U\paren{\set X}:=\bigcup_{x\in\set X} \{p_1(x),p_2(x)\}$ --- that is, it `unpacks' a set of pairs of operators into a set of operators;  in an abuse of notation, we shall also allow $\set U$ to apply to sequences, so that $\set U(\lst P) := \set U\paren{\{\lst P_i\}_i}$, and to individual pairs, so that if $X$ is a single pair then $\set U(X) := \set U(\{X\})$;
\item finally, a \emph{choice of $N$ qubits stabilized by $\set S$}, $\lst P$, is a sequence of $N$ pairs of operators from the Pauli group such that 
\begin{enumerate}
\item $\set U(\lst P)\cap \set S = \emptyset$ --- that is, the operators in $\lst P$ are not included in $\set S$;
\item every pair in $\lst P$ is a conjugal pair with respect $\set S \cup \set U(\lst P)$;
\item $(\om_{\set S}\circ p_1)(\lst P_i)=m_{\set S}(\lst P_i)$; and
\item $\lst M^{(N)}(\lst P)$ is an ordered sequence.
\end{enumerate}
\end{itemize}

\end{definition}
%@nonl
%@-node:gcross.20091221145013.1277:Definition: Additional notation
%@+node:gcross.20091221145013.1278:Definition: Total ordering => Optimal choice of qubits
Given the notation above, we now precisely define what we mean by the ``best choice'' of logical qubits.

\begin{definition}
An \emph{optimal choice of $N$ qubits stabilized by $\set S$} is any choice of $N$ qubits, $\lst P$ stabilized by $\set S$, such that given any other choice of $N$ qubits, $\lst P'$, that is also stabilized by $\set S$, such that $(\genfun\circ \set U)(\lst P)=(\genfun\circ \set U)(\lst P')$, we have that $\lst M^{(N)}(\lst P)_i \ge \lst M^{(N)}(\lst P')_i$ for all $1\le i \le N$.
\end{definition}
%@-node:gcross.20091221145013.1278:Definition: Total ordering => Optimal choice of qubits
%@+node:gcross.20100128212359.1294:Definition: Unimprovable set
How do we know that a choice of qubits is optimal?  It turns out that there is a very simple condition that is sufficient for demonstrating this.  First, we need the following definition:

\begin{definition}
An \emph{unimprovable set with respect to $\set S$} is a set of Pauli operators, $\set O$, such that for any subset, $\set X\subseteq \set O$, we have that $\om_{\set S}\paren{\prod_{x\in \set X} x} = \min_{x\in\set X}\om_{\set S}(x)$.  We say that an unimprovable set $\set O$ \emph{extends to $\lst Q$} if for all subsets $\set X \subseteq \set O\cup\set Q$ such that $\set X\cap \set O \ne \emptyset$ we have that $\om_{\set S}\paren{\prod_{x\in \set X} x} \le \min_{x\in\set X\cap \set O}\om_{\set S}(x)$.
\end{definition}
%@-node:gcross.20100128212359.1294:Definition: Unimprovable set
%@+node:gcross.20100125183648.1308:Theorem: Optimality condition
%@+node:gcross.20100125183648.1309:Statement
\begin{theorem}
\label{optimality-condition}
If $\lst P$ is a choice of $N$ logical qubits stabilized by $\set S$ such that $\{p_1(\lst P_i)\}_i\}$ is an unimprovable set with respect to $\set S$ that extends to $\set U(\lst P)$, then $\lst P$ is an optimal choice of qubits.
\end{theorem}

Before we can prove this Theorem, we need a couple of Lemmas and a Proposition.
%@-node:gcross.20100125183648.1309:Statement
%@+node:gcross.20100125183648.1306:Lemma: Combinations can't make things worse
\begin{lemma}
\label{combinations-can't-make-things-worse}
For any set of operators $\set O$, we have that $\om_{\set S}\paren{\prod_{o\in\set O} o} \ge \text{min}_{o\in \set O}\,\om_{\set S}\paren{o}$.
\end{lemma}

\begin{proof}[Proof of Lemma]
Any undetectable error with respect to $\set S$ acting on $\om_{\set S}\paren{\prod_{o\in\set O} o}$ must also act at least one of the operators in $\set O$ since otherwise it cannot anti-commute with the product. \end{proof}
%@-node:gcross.20100125183648.1306:Lemma: Combinations can't make things worse
%@+node:gcross.20100125183648.1304:Lemma: The lesser operator wins
\begin{lemma}
\label{lesser-operator-wins}
Suppose we are given two operators $a,b\in\pauligroup$ such that $\om_{\set S}(a)<\om_{\set S}(b)$;  then $\om_{\set S}(a\cdot b) = \om_{\set S}(a)$.
\end{lemma}

\begin{proof}[Proof of Lemma]
Since $\om_{\set S}(a)<\om_{\set S}(b)$, there must be an undetectable error with respect to $\set S$ that acts on $a$ but not on $b$;  thus, it must anti-commute with and hence act on the product $a\cdot b$, so that $\om_{\set S}(a\cdot b)\le \om_{\set S}(a)$.  Since $\om_{\set S}(a\cdot b)\ge \om_{\set S}(a)$ by Lemma , we conclude that $\om_{\set S}(a\cdot b) = \om_{\set S}(a)$.
\end{proof}
%@nonl
%@-node:gcross.20100125183648.1304:Lemma: The lesser operator wins
%@+node:gcross.20100125183648.1319:Proposition: Bound on recombinations
%@+node:gcross.20100125183648.1320:Statement
\begin{proposition}
\label{bound-on-recombinations}
Suppose we are given
\begin{enumerate}
\item sets of Pauli operators $\set Q$ and $\set S$;
\item a non-empty set of conjugal pairs, $\set C$, with respect to $\set U(\set C) \cup \set S$, such that $U(\set C)\subseteq \genfun(\set Q)$; and \item a set $\set A$ with the property that for any conjugal pair $X:=(a,b)$ such that $\{a,b\}\in\genfun(\set C)$, we must have that $\set G_{\set Q}(X) \cap \set A \ne \emptyset$.
\end{enumerate}
Then $|\set C|\le|\set A|$.
\end{proposition}

In order to prove this Proposition, we will first need some additional Lemmas.
%@-node:gcross.20100125183648.1320:Statement
%@+node:gcross.20100125183648.1321:Single pair rearrangement.
\begin{lemma}
\label{single-pair-rearrangement}
Let $A:=(a,b)$ with $\{a,b\}\subseteq\genfun(\set Q)$ be a conjugal pair with respect to some set $\set S$, and $o$ be some Pauli operator such that $o\in\set G_{\set Q}(A)$.  Then there exists a pair $B:=(c,d)$ such that
\begin{enumerate}
\item $\{c,d\}\subseteq\genfun(\set Q)$;
\item $o\in \set G_{\set Q}(c)$;
\item $o\notin \set G_{\set Q}(d)$;
\item $(c,d)$ is a conjugal pair with respect to $\paren{\set S\backslash \{a,b\}}\cup\{c,d\}$; and
\item $(\genfun\circ\set U)(B) = (\genfun\circ\set U)(A)$.
\end{enumerate}
\end{lemma}

\begin{proof}
Let
$$
(c,d):=
\begin{cases}
\paren{a,b} & o\in \set G_{\set Q}(a), o\notin \set G_{\set Q}(b) \\
\paren{b,a} &  o\notin \set G_{\set Q}(a), o\in \set G_{\set Q}(b) \\
\paren{a,b\cdot a} & o\in \set G_{\set Q}(a), o\in \set G_{\set Q}(b) \\
\end{cases}
$$
Note that in any of the above cases, properties 1-3 are satisfied by construction, property 4 is satisfied because $c$ and $d$ are products of $a$ and $b$ which commute with every element in $\set S\backslash \{a,b\}$ and $\{c,d\}=0$, and finally property 5 is satisfied because $\{a,b\}\subseteq \genfun\paren{\{c,d\}}$ and $\{c,d\}\subseteq \genfun\paren{\{a,b\}}$.
\end{proof}
%@nonl
%@-node:gcross.20100125183648.1321:Single pair rearrangement.
%@+node:gcross.20100125183648.1322:Directed Gaussian elimination
\begin{lemma}
\label{directed-gaussian-elimination-of-logicals}
In the context of Proposition \ref{bound-on-recombinations}, suppose we are given an element $a\in \set A$ with the property that there exists a pair $Y''\in\set C$ such that $a\in\set G(Y'')$. Then there exists a conjugal pair $Y$ and set of conjugal pairs $\set D$, all with respect to $\set U\paren{\{Y\}\cup \set D} \cup \set S$, such that
\begin{enumerate}
\item $|\set D| = |\set C|-1$
\item $(\genfun\circ\set U)(\{Y\}\cup \set D)=(\genfun\circ\set U)(\set C)$;
\item $a\in (\set G_{\set Q}\circ p_1)(Y)$ but $a\notin (\set G_{\set Q}\circ p_2)(Y)$;
\item $a\notin \bigcup_{D\in \set D} \set G_{\set Q}(D)$; and
\item for every conjugal pair $O\in\set D$, we have that $\set G_{\set Q}(O) \cap \set A\backslash \{a\} \ne \emptyset$.
\end{enumerate}
\end{lemma}

\begin{proof}
Proof by induction on the size of $\set C$.  If $\set C=\{Y''\}$, then apply Lemma \ref{single-pair-rearrangement} letting $o:=a$, $A:=Y''$, and $Y:=B$, and we see that we have a pair $Y$ which is conjugal with respect to $\{Y\}\cup\set S$ and also such that $(\genfun\circ\set U)(\{Y\})=(\genfun\circ\set U)(\{Y''\})$.  Let $\set D:=\emptyset$, and we see that the remaining properties hold trivially, so we are done.

Now let us assume that this lemma has been proven for the case where $|\set C|=n-1$, and we are given a set $\set C$ with $n$ elements.  Take any $X''\in\set C\backslash\{Y''\}$, and apply the lemma to $\set C\backslash \{X''\}$, $\set A$, $a$ and $Y''$ to obtain the objects $Y'$ and $\set D'$ described in this Lemma without the primes.  If $a\notin\set G(X'')$, then by the assumptions of the Lemma we know that $\set G(X'') \cap \set A\backslash \{a\} \ne \emptyset$, so let $Y:=Y'$ and $\set D:=\set D'\cup\{X''\}$, and we are done.

Otherwise, apply Lemma \ref{single-pair-rearrangement}, setting $A:=X''$, $o:=a$, and $X':=B$, and let $X:=\paren{p_1(X')\cdot p_1(Y'),p_2(X')}$ and $Y:=\paren{p_1(Y'),p_2(Y')\cdot p_2(X')}$.  Note $X'$ and $Y'$ are conjugal pairs with respect to $\set U\paren{\{X',Y'\}\cup\set D}\cup\set S$ and $\{X',Y'\}\cap \paren{\set D\cup\set S}=\emptyset$, and so by Lemma \ref{combining-pairs} we conclude that $X$ and $Y$ are conjugal pairs with respect to $\set U\paren{\{X,Y\}\cup\set D}\cup\set S$, and also that $(\genfun\circ\set U)(\{X,Y\})=(\genfun\circ\set U)(\{X',Y'\})$;  since $X'$ was obtained from applying Lemma \ref{single-pair-rearrangement} to $X''$ and $a$, we furthermore conclude that $(\genfun\circ\set U)(\{X,Y\})=(\genfun\circ\set U)(\{X'',Y'\})$.  Since $X'$ was obtained as a result of Lemma \ref{single-pair-rearrangement}, we know that $a\in (\set G \circ p_1)(X')$ but $a\notin (\set G_{\set Q} \circ p_2)(X')$, and we also know from the earlier recursive application of this Lemma that $a\in (\set G \circ p_1)(Y')$ but $a\notin (\set G_{\set Q} \circ p_2)(Y')$.  Thus, we observe that by construction, $a\in (\set G_{\set Q} \circ p_1)(Y)$, and $a\notin \paren{(\set G_{\set Q} \circ p_2)(Y) \cup \set G_{\set Q}(X)}$.

Let $\set D:=\{X\}\cup\set D'$, and observe that $|\set D|=|\set D'|+1=|\set C\backslash \{X''\}|-1+1=|\set C|-1$, and also that $(\genfun\circ\set U)(\{Y\}\cup\set D)=(\genfun\circ\set U)(\{X,Y\}\cup\set D')=(\genfun\circ\set U)(\{X''\}\cup(\{Y'\}\cup\set D'))=(\genfun\circ\set U)(\{X''\}\cup(\set C'\backslash\{X''\}))=(\genfun\circ\set U)(\set C)$.  Furthermore, by the earlier recursive application of this Lemma we know that $a\notin\set G_{\set Q}(O)$ for every $O\in \set D'$, so since we have also established that $a\notin \set G_{\set Q}(X)$, we conclude that $a\notin\set G_{\set Q}(O)$ for every $O\in \set D$;  since also know that every such $O$ must also satisfy $\set G_{\set Q}(O)\cap \set A \ne \emptyset$, we conclude that every such $O$ satisfies $\set G_{\set Q}(O) \cap A\backslash\{a\}\ne\emptyset$.
\end{proof}
%@-node:gcross.20100125183648.1322:Directed Gaussian elimination
%@+node:gcross.20100125183648.1323:Undirected Gaussian elimination
\begin{lemma}
\label{undirected-gaussian-elimination-of-logicals}
In the context of Proposition \ref{bound-on-recombinations}, there exists a Pauli operator $a$ satisfying the assumption of Lemma \ref{directed-gaussian-elimination-of-logicals}.
\end{lemma}

\begin{proof}
Take any pair $Y'\in\set C$.  By the assumptions of Proposition \ref{bound-on-recombinations}, we know that $\set G(Y')\cap \set A \ne \emptyset$, which implies that there exists an element $a\in A$ such that either $a\in (\set G_{\set Q}\circ p_1)(Y')$ or $a\in (\set G_{\set Q}\circ p_2)(Y')$.  The existance of $Y$ and $\set D$ then follow immediately from the application of Lemma \ref{directed-gaussian-elimination-of-logicals}.
\end{proof}
%@nonl
%@-node:gcross.20100125183648.1323:Undirected Gaussian elimination
%@+node:gcross.20100125183648.1324:Elimination to create subset
\begin{lemma}
\label{elimination-to-create-subset}
In the context of Proposition \ref{bound-on-recombinations}, there exists a set of conjugal pairs, $\set X$, with respect to $\set X\cup\set S$, and a subset of operators, $\set O\subseteq \set A$, such that
\begin{enumerate}
\item $|\set X|=|\set O|=|\set C|$;
\item $(\genfun\circ\set U)(\set X)=(\genfun\circ\set U)(\set C)$; and
\item for every $o\in\set O$, there is a conjugal pair $Y\in\set X$ such that $o\in\set G_{\set Q}(Y)$;
\end{enumerate}
\end{lemma}

\begin{proof}
Proof by induction.  If $\set C$ is empty, then the empty sets trivially satisfy this Lemma.

Now suppose that we have proven this Lemma for $|\set C|=N-1$, and assume we have been given sets $\set C$ and $\set A$ such that $|\set C|=N$.  Applying Lemma \ref{undirected-gaussian-elimination-of-logicals} to $\set C$ and $\set A$ we obtain the conjugal pair $Y$,  the set of conjugal pairs $\set D$, and the element $a$ described in the conclusions of that Lemma.  Apply this Lemma recursively to the respective sets $\set D$ and $\set A\backslash\{a\}$, we obtain the sets $\set X'$ and $\set O'$ described (without the primes) in this Lemma; let $\set X := \set X'\cup\{Y\}$ and $\set O:=\set O'\cup\{a\}$.  Note that $\set X$ is a set of conjugal pairs with respect to $\set X\cup\set S$ since $\set X'$ is a set of conjugal pairs with respect to $\set X'\cup\set S$, and we know that the operators in $Y$ commute with every operator in every pair in $\set X'$ since they commute with every operator in $(\genfun\circ\set U)(\set D)=(\genfun\circ\set U)(\set X')$.

First, observe that $|\set O|=|\set O'|+1$ since $a\notin \set O'$.  Futhermore, $a\notin \bigcup_{x\in \set U(\set X')} \set G(x)$ since $a\notin \bigcup_{x\in \set U(\set D)} \set G_{\set Q}(x)$ by Lemma \ref{undirected-gaussian-elimination-of-logicals} and $(\genfun\circ\set U)(\set D)=(\genfun\circ\set U)(\set X')$ by recursive application of this Corolary.  Thus, $|\set X|=|\set X'|+1$ since $Y\notin X'$ as $a\in\set G(Y)$, and $|\set X|=|\set O|=|\set D|+1=|\set C|-1+1=|\set C|$.

Second, observe that since $(\genfun\circ\set U)(\set X')=(\genfun\circ\set U)(\set D)=(\genfun\circ\set U)(\set C)$ by recursive application of this Lemma and $(\genfun\circ\set U)(\{Y\}\cup\set D)=(\genfun\circ\set U)(\set C)$ by Lemma \ref{undirected-gaussian-elimination-of-logicals}, we conclude that $(\genfun\circ\set U)(\set X) = (\genfun\circ\set U)(\{Y\}\cup\set X') = (\genfun\circ\set U)(\{Y\}\cup\set D) = (\genfun\circ\set U)(\set C)$.

Finally, observe that for every $o\in\set O$ we either have that $o=a$, in which case $Y\in\set X$ and $o\in\set G_{\set Q}(Y)$, or $o\in \set A\backslash\{a\}$, in which case by recursive application of this Lemma we know that there is an operator $Z\in \set X'\subseteq \set X$ such that $o\in\set G_{\set Q}(Z)$.
\end{proof}
%@nonl
%@-node:gcross.20100125183648.1324:Elimination to create subset
%@+node:gcross.20100125183648.1325:Proof
With these Lemmas having performed the heavy lifting, the Proof of Proposition {bound-on-recombinations} is quite simple.

\begin{proof}[Proof of Proposition \ref{bound-on-recombinations}]
Proof by contradiction.  By Lemma \ref{elimination-to-create-subset}, there would have to exist a subset $\set O\subseteq\set A$ such that $|\set C|=|\set O|>|\set A|$, which is impossible.
\end{proof}
%@-node:gcross.20100125183648.1325:Proof
%@-node:gcross.20100125183648.1319:Proposition: Bound on recombinations
%@+node:gcross.20100125183648.1326:Proof
We now have all of the tools that we need to prove Theorem \ref{optimality-condition}.

\begin{proof}[Proof of Theorem \ref{optimality-condition}]
Proof by contradiction.  Let $\lst P'$ be some choice of qubits stabilized by $\set S$ such that $(\genfun\circ\set U)(\lst P)=(\genfun\circ\set U)(\lst P')$ and there exists some integer $k$ such that $M^{(N)}_{\set S}(\lst P')_k > M^{(N)}_{\set S}(\lst P)_k$;  in particular, let $k$ be the smallest such integer, and let $\set C:=\{\lst P'_i : i \ge k\}$.  Let $l$ be the smallest integer such that $M^{(N)}_{\set S}(\lst P)_l\ge M^{(N)}_{\set S}(\lst P')_k$ or $N+1$ if there is no such integer, and let $\set A := \{p_1(\lst P_i) : i \ge l\}$; note that since $M^{(N)}(\lst P')_k > M^{(N)}(\lst P)_k$ we must have $l>k$, and hence $|\set C| > |\set A|$.

Take any pair $O:=(a,b)$ such that $\{a,b\}\in\genfun(\set C)$.  Since $a$ and $b$ anti-commute, it must be the case that $\{p_1(\lst P_i)\}_i\cap \set G_{\set U(\lst P)}(O)\ne\emptyset$, because if every operator in $\set G_{\set U(\lst P)}(O)$ were the second member of a pair in $\lst P$ then $a$ and $b$ would commute.  Let $c$ be a choice of $a$ or $b$ such that $\{p_1(\lst P_i)\}_i\cap \set G_{\set U(\lst P)}(c)\ne\emptyset$.  By Lemma \ref{combinations-can't-make-things-worse} we know that $M^{(N)}_{\set S}(\lst P')_k\le\om_{\set S}(c)$ since $c\in\genfun(\set C)$.  By the assumption of this Theorem that $\{p_1(\lst P_i)\}_i$ is an unimprovable set that extends to $\set U(\lst P)$, we know that $M^{(N)}_{\set S}(\lst P')_k \le \om_{\set S}(c)\le\min_{x\in\{p_1(\lst P_i)\}_i\cap \set G_{\set U(\lst P)}(c)} \om_{\set S}(x)$.  From these bounds we conclude that $\{p_1(\lst P_i)\}_i\cap \set G_{\set U(\lst P)}(c)\subseteq \set A$, and since $\{p_1(\lst P_i)\}_i\cap \set G_{\set U(\lst P)}(c)\ne\emptyset$ we see therefore that $\set G_{\set U(\lst P)}(c)\cap\set A\ne\emptyset$ and so $\set G_{\set U(\lst P)}(O)\cap\set A\ne\emptyset$.

We have now demonstrated that for every pair $O:=(a,b)$ such that $\{a,b\}\in\set \genfun(\set C)$, we must have $\set G(O)\cap\set A \ne\emptyset$.  Observe that this means that sets $\set C$ and $\set A$ match the descriptions in Proposition \ref{bound-on-recombinations} (letting set $\set Q:=\{\set P_i\}_i$), and thus we see that it is impossible for $|\set C|>|\set A|$, and so we have a contradiction.  We thus conclude that no such choice $\lst P'$ can exist.
\end{proof}
%@nonl
%@-node:gcross.20100125183648.1326:Proof
%@-node:gcross.20100125183648.1308:Theorem: Optimality condition
%@+node:gcross.20100126164527.1288:Lemma: Multiple recombinations
\begin{lemma}
\label{multiple-recombinations}
Suppose we are given
\begin{itemize}
\item a pair of Pauli operators, $(a,b)$;
\item a sequence of $N$ pairs of Pauli operators, $\lst R$;
\item a set of Pauli operators, $\set S$;
\item and two sequences of elements from $\{0,1\}$, $\lst f^{(1)}$ and $\lst f^{(2)}$;
\end{itemize}
such that $(a,b)$ and the pairs in $\lst R$ are all conjugal to $\{a,b\}\cup\set U(\lst R)\cup\set S$.  Define
$$b':=b\cdot\prod_{1 \le i \le N, k \in \{0,1\}, f^{(k)}_i=1}p_{(3-k)}(\lst R_i),$$
$\lst O'_i := (o'^{(1)}_i,o'^{(2)}_i)$ where
$$o'^{(k)}_i :=
\begin{cases}
p_k(\lst O_i) & l^{(k)}_i = 0\\
p_k(\lst O_i) \cdot a & l^{(k)}_i = 1,
\end{cases}
$$
and
$$\set S' := \left[\set S\backslash\paren{\{a,b\} \cup \set U(\lst R)}\right]\cup \{a,b'\} \cup \set U(\lst R').$$
Then
\begin{enumerate}
\item the pair $(a,b')$ and the sequence of pairs $\lst R'$ are all conjugal pairs with respect to $\set S'$; and
\item $\genfun\paren{\{a,b'\}\cup \set U(\lst R')}=\genfun\paren{\{a,b\}\cup \set U(\lst R)}$
\end{enumerate}
\end{lemma}

\begin{proof}
Proof by induction.  For the base case, note that statement trivially holds for $N=0$.

Now suppose we have proven this for all sequences of $N-1$ pairs, and we are given a sequence of $N$ pairs.  Invoking the inductive hypothesis, apply this Lemma recursively to the first $N-1$ elements of $\lst R$ and $\lst f$, and let $b''$, $\lst R''$ and $\set S''$ refer to the objects described in this Lemma (without the extra primes) as applied to these $N-1$ elements.

Let
$$
c := 
\begin{cases}
b'' & l^{(1)}_N = 0\\
b'' \cdot p_2(\lst R_N)  &  l^{(1)}_N = 1,\\
\end{cases}
\quad
d_1 :=
\begin{cases}
p_1(\lst R_N) & l^{(1)}_N = 0\\
p_1(\lst R_N) \cdot a &  l^{(1)}_N = 1,\\
\end{cases}
$$
and
$$\set T := \left[\set S''\backslash\{a,b'',p_1(\lst R_N),p_2(\lst R_N)\}\right]\cup \{a,c,d_1,p_2(\lst R_N)\}.$$
Note that for either value of $f^{(1)}$ we have that the pairs $(a,c)$ and $(d_1,p_2(\lst R_N))$ are conjugal with respect to $\set R$ and that $\genfun\paren{\{a,c,d_1,p_2(\lst R_N)\}}=\genfun\paren{\{a,b'',p_1(\lst R_N),p_2(\lst R_N)\}}$; for $f^{(1)}=0$ this conclusion is trivial since $b''=c$ and $d_1=p_1(\lst R_N)$, and for $f^{(1)}=1$ we obtain this conclusion by applying Lemma \ref{combining-pairs}.

Now let
$$
c' := 
\begin{cases}
c & l^{(2)}_N = 0\\
c \cdot d_1  &  f^{(2)}_N = 1,\\
\end{cases}
\quad
d_2 :=
\begin{cases}
p_2(\lst R_N) & f^{(2)}_N = 0\\
p_2(\lst R_N) \cdot a &  f^{(2)}_N = 1,\\
\end{cases}
$$
and
$$\set T' := \left[\set T\backslash\{a,c,d_1,p_2(\lst P_N)\}\right]\cup \{a,c',d_1,d_2\}.$$
Applying the same reasoning as before, we conclude that the pairs $(a,c')$ and $(d_1,d_2)$ are conjugal with respect to $\set R'$ and that $\genfun\paren{\{a,c',d_1,d_2\}}=\genfun\paren{\{a,c,d_1,p_2(\lst P_N)\}}$.

Finally, let
$$c'' :=
\begin{cases}
c' \cdot a & f^{(1)}_N = 1 \,\,\text{and}\,\, f^{(2)}_N = 1 \\
c' & \text{otherwise},\\
\end{cases}
$$
and
$$
\set T'':=
\left[\set T'\backslash\{a,c'\}\right]\cup \{a,c''\}.
$$
Observe that $\{a,c''\}=0$ and that $a$ and $c''$ are products of $a$ and $c$;  thus, $\genfun\paren{\{a,c'',d_1,d_2\}}=\genfun\paren{\{a,c',d_1,d_2\}}$, and furthermore $a$ and $c''$ commute with every operator in $\set T''$ save each other and so $\{(a,c'')\} \cup \{(d_1,d_2\}$ are conjugal pairs with respect to $\set T''$.

Note, however, that by construction we have that $b'=c''$, $\set S'=\set T''$, and
$$\lst R'_i :=
\begin{cases}
\lst R''_i & i < N \\
(d_1,d_2) & i = N.\\
\end{cases}
$$
Thus, we are done.
\end{proof}
%@nonl
%@-node:gcross.20100126164527.1288:Lemma: Multiple recombinations
%@+node:gcross.20100127132544.1287:Corolary: Recombination to make commute
\begin{corolary}
\label{recombination-to-make-commute}
Suppose we are given
\begin{itemize}
\item a pair of Pauli operators, $(a,b)$;
\item a sequence of $N$ pairs of Pauli operators, $\lst R$;
\item a set of Pauli operators, $\set S$;
\item and a Pauli operator $x$;
\end{itemize}
such that $(a,b)$ and the pairs in $\lst R$ are all conjugal to $\{a,b\}\cup\set U(\lst R)\cup\set S$, and $\{a,x\}=0$.  Let $b' := b\cdot \prod_{X\in\set R, \{p_k(X),a\}=0} p_{(3-k)}(X)$, $\set R':=\{\paren{f(c),f(d)}: (c,d)\in\set R\}$ where
$$
f(o) :=
\begin{cases}
o & [o,x]=0 \\
o\cdot a & \{o,x\}=0, \\
\end{cases}
$$
and
$$\set S' := \left[\set S\backslash\paren{\{a,b\} \cup \set U(\set R)}\right]\cup \{a,b'\} \cup \set U(\set R').$$
Then
\begin{enumerate}
\item the pair $(a,b')$ and the set of pairs $\set R'$ are all conjugal pairs with respect to $\set S'$;
\item $\genfun\paren{\{a,b'\}\cup \set U(\set R')}=\genfun\paren{\{a,b\}\cup \set U(\set R)}$;
\item $x$ commutes with every operator in $\set R'$;
\item and $$\min_{r'\in \set U(\set R')}\om_{\set S}(r')\ge \min\paren{\om_{\set S}(a),\min_{r\in \set U(\set R)}\om_{\set S}(r)};$$
\end{enumerate}
\end{corolary}

\begin{proof}
Place any arbitrary ordering in $\set R$, and the define then sequences $f^{(1)}$ and $f^{(2)}$ such that
$$\lst f^{(j)}_i :=
\begin{cases}
0 & [p_j(\lst R_i),x] = 0 \\
1 & \{p_j(\lst R_i),x\} = 0. \\
\end{cases}
$$
The first two conclusions of this Corolary follows immediately from Lemma \ref{multiple-recombinations}, dropping the ordering on $\lst R'$.  The third comes from the fact that every element in $\set R'$ is either an element from $\set R$ that commutes with $x$ or the product of $a$ with an element from $\set R$ that anti-commutes with $x$, which must also commute with $x$.  The fourth comes directly from Lemma \ref{combinations-can't-make-things-worse}.
\end{proof}
%@-node:gcross.20100127132544.1287:Corolary: Recombination to make commute
%@+node:gcross.20100129120448.1279:Lemma: Replacing element with product preserves unimprovability
\begin{lemma}
\label{replacing element with product preserves unimprovability}
Suppose we are given an unimprovable set $\set O$ with respect to $\set S$.  Let $o$ be any element in $\set O$, and $\set Y\subseteq \set O$ such that $o\in y$ and $\min_{y\in\set Y}\om_{\set S}(y)=\om_{\set S}(o)$.  Let $o':=\prod_{y\in\set Y}y$ and $\set O' := \paren{\set O\backslash\{o\}}\cup\{o'\}$.  Then
\begin{enumerate}
\item $\om_{\set S}(o)=\om_{\set S}(o')$;
\item $\genfun(\set O)=\genfun(\set O')$;
\item and $\set O'$ is also an unimprovable set with respect to $\set S$;
\end{enumerate}
\end{lemma}

\begin{proof}
First observe that since $\set O$ is an unimprovable set and $\min_{y\in\set Y}\om_{\set S}(y)=\om_{\set S}(o)$, we conclude that $\om_{\set S}(o)=\om_{\set S}(o')$, which proves conclusion 1.

The only element changed from $\set O$ to $\set O'$ is the replacement of $o$ with a product of elements including $o$;  since every factor besides $o$ in the new product is also contained in $\set O'$, we conclude that we can obtain $o$ from a product of elements in $\set O'$, and so $\genfun(\set O)=\genfun(\set O')$, which proves conclusion 2.

Take any subset of elements $\set X'\subseteq\set O'$, and let $x' := \prod_{y\in \set X'} y$ and $d_{x'}:=\min_{y\in\set X'}\om_{\set S}(y)$.  We need to show that $\om_{\set S}(x') = d_{x'}$.  If $o'\notin\set X'$, then this follows from the fact that $\set X'\subseteq\set O$ and $\set O$ is an unimprovable set.

Suppose that $o'\in\set X'$. Since we showed in the first paragraph that $\genfun(\set O)=\genfun(\set O')$, we conclude that there exists a subset $\set X\subseteq\set O$ such that $x=\prod_{y\in \set X'}y$.  For every $a\in\set X'$ such that $\om_{\set S}(a)<\om_{\set S}(o')$, we know that $a\in\set X$ because $a\notin\set Y$ since $\om_{\set S}(o)=\om_{\set S}(o')$ and $\set O$ is an unimprovable set.  Thus, if there exists at least one such $a$, then $\min_{y\in\set X'}\om_{\set S}(y)=\min_{y\in\set X}\om_{\set S}(y)=\om_{\set S}(x')$.  If there is no such $a$, then $\min_{y\in\set X'}\om_{\set S}(y)=\om_{\set S}(o')$, and so $\min_{y\in\set X'}\om_{\set S}(y)=\om_{\set S}(o)=\min_{y\in\set X'}\om_{\set S}(y)$.  We have thus proven that $\set O'$ is an unimprovable set, which proves the last conclusion.

\end{proof}
%@nonl
%@-node:gcross.20100129120448.1279:Lemma: Replacing element with product preserves unimprovability
%@+node:gcross.20100129120448.1281:Lemma: Recombining elements preserves extension
\begin{lemma}
\label{recombining elements preserves extension}
Suppose $\set O$ and $\set O'$ are unimprovable sets with respect to $\set S$ such that $\genfun(\set O)=\genfun(\set O')$.  Then if $\set O$ extends to $\set Q$, so does $\set O'$.
\end{lemma}

\begin{proof}
Take any subset $\set X \subseteq \set O\cup\set Q$ such that $\set A' := \set X\cap \set O' \ne \emptyset$.  We need to show that $\om_{\set S}(x)\le\min_{y\in\set A'}\om_{\set S}(y)$, where $x := \prod_{y\in\set X} y$.  Let $\set B := \set X \backslash \set A'$, $a := \prod_{y\in\set A'} y$, $b := \prod_{y\in\set B} y$ and $\set A$ be the subset of $\set O$ such that $\prod_{y\in\set A} y =a$.  Since $\set O$ and $\set O'$ are unimprovable sets, we know that $\min_{y\in\set A'}\om_{\set S}(y)=\om_{\set S}(a)=\min_{y\in\set A}\om_{\set S}(y)$.  Since $x=a\cdot b$, we see that $x$ is equal to the product of a set of elements $\set C := \set A \cup\set B$ such that $B\subseteq \set Q$ and $\set C\cap\set O=\set A$, and since $\set O$ extends to $\set Q$ we conclude that $\om_{\set S}(x)\le\min_{y\in\set A}\om_{\set S}(y)=\min_{y\in\set A'}\om_{\set S}(y)$.  Thus, $\set O'$ extends to $\set Q$.
\end{proof}
%@-node:gcross.20100129120448.1281:Lemma: Recombining elements preserves extension
%@+node:gcross.20100129120448.1283:Lemma: Recombining extension elements preserves extension
\begin{lemma}
\label{recombining extension elements preserves extension}
If $\set O$ is an unimprovable set with respect to $\set S$ that extends to $\set Q$, and $\set Q'$ is a set such that $\set Q'\subseteq\genfun(\set Q)$, then $\set O$ also extends to $\set Q'$.
\end{lemma}

\begin{proof}
Take any subset $\set X' \subseteq \set O\cup\set Q'$ such that $\set A := \set X'\cap \set O \ne \emptyset$.  We need to show that $\om_{\set S}(x)\le\min_{y\in\set A}\om_{\set S}(y)$, where $x := \prod_{y\in\set X} y$.  Note that since $\set Q'\subseteq\genfun(\set Q)$, there exists a set $\set B\subseteq \set Q$ such that $x = \prod_{y\in\set A\cup\set B} y$, and so since $\set O$ extends to $\set Q$ we conclude that $\om_{\set S}(x)\le\min_{y\in\set A}\om_{\set S}(y)$.
\end{proof}
%@-node:gcross.20100129120448.1283:Lemma: Recombining extension elements preserves extension
%@+node:gcross.20100203143542.1282:Lemma: Recombining elements preserves conjugal property for third parties
\begin{lemma}
\label{recombining elements preserves conjugal property for third parties}
Suppose we are given two sets, $\set A$ and $\set B$, of pairs of Paili operators that are conjugal to $\set U(\set A) \cup \set U(\set B) \cup \set S$ for some set $\set S$.  Then if $(\genfun\circ \set U)(\set B')\subseteq(\genfun\circ \set U)(\set B)$ for some set $\set B'$, then $\set A$ is also conjugal to $\set A \cup \set B' \set S$.
\end{lemma}

\begin{proof}
Since every operator in $\set U(\set A)$ commutes with every operator in $\set U(\set B)$, they must also commute with every operator in $\set U(\set B')$.
\end{proof}
%@-node:gcross.20100203143542.1282:Lemma: Recombining elements preserves conjugal property for third parties
%@+node:gcross.20100127211423.1361:Theorem: Creation of certificate
\begin{theorem}
Suppose we are given a set of $N$ pairs $\set L$ and a set of Pauli operators $\set S$ such that every pair in $\set L$ is conjugal with respect to $\set U(\set L)\cup\set S$.  Then for every $0\le n \le 2N$, there exist
\begin{itemize}
\item a choice of qubits $\lst P$ stabilized by $\set S$;
\item a set of pairs, $\set Q$;
\item and an unimproveable set of operators $\set O$ with respect to $\set S$ that extends to $\set X:=\set U(\set Q) \cup \paren{\set U(\lst P) \backslash \set O}$;
\end{itemize}
such that
\begin{enumerate}
% 1
\item \label{pairs are all conjugal to T} the pairs in $\lst P$ and $\set Q$ are all conjugal with respect to $\set T':=\set U(\lst P)\cup \set U(\set Q)\cup \set S$;
% 2
\item \label{P and Q generate the same set as L} $\genfun\paren{\set U(\lst P)\cup \set U(\set Q)}=(\genfun\circ\set U)(\set L)$;
% 3
\item \label{first members are all in O} $p_1(\lst P_i)\in \set O$ for every $1 \le i \le |\lst P_i|$;
% 4
\item \label{size of O matches arbitrarily chosen integer} $|\set O|=n$;
% 5
\item \label{operators in X are no worse than the best qubit in P} and $\min_{x\in\set X}\om_{\set S}(x)\ge m_{\set S}(\lst P_{|\lst P|})$ (if $\lst P$ is non-empty).
\end{enumerate}
\end{theorem}

\begin{proof}

Proof by induction.  For the base case, let $\lst P$ be the empty sequence, $\set Q:=\set L$, and $\set O:=\emptyset$;  then the conclusions are satisfied vacuously.

Now assume that this Proposition has been proven for some integer $n-1$;  let $\lst P'$, $\lst Q'$, $\set O'$ and $\set X'$ be the named objects (without the primes) obtained by recursively applying this Theorem to the integer $n-1$  Let $h$ be any minimizer of $w$ over the set $\bigcup_{o\in\set X'} \set e_{\set S}(o)$.  Let $x$ be either $p_2(\lst P_k)$ for the smallest $k$ such that $\{p_2(\lst P_k),h\}=0$, or if there is no such $k$ then let $x$ be any arbitrarily chosen operator from $\set X'$ such that $\{x,h\}=0$.  Finally, let  $d:=\om_{\set S}(x)$.

There are two cases;  either
\begin{enumerate}
\item $x\in \set U(\lst P')$;
\item or $x\notin \set U(\lst P')$ and $o\in\set U(\set Q')$.
\end{enumerate}

\begin{description}

\item[Case 1 $\Rightarrow x\in\set U(\lst P')$]
Let $k$ be the integer such that $x=p_2(\lst P'_k)$, $a := x$, and $b:=p_1(\lst P'_k)$.  Now let $\set Q:=\set R'$ and $b'$ be the objects named in Corolary \ref{recombination-to-make-commute} given the pair $(a,b)$, the set $\set R=\set Q'$, the set $\set S$, and the operator $x:=h$.  Note that by Corolary \ref{recombination-to-make-commute} we know that $(a,b')$ and every pair in $\set Q$ are conjugal to
$$\left[\set T'\backslash\paren{\{a,b\} \cup \set U(\set Q')}\right]\cup \{a,b'\} \cup \set U(\set Q)=\set T''$$
where
$$\set T'':=\left[\paren{\set U(\lst P')\cup \set S}\backslash\{b\}\right]\cup \{b'\} \cup \set U(\set Q);$$
we also know that $\genfun\paren{\{b'\}\cup \set U(\set Q)}=\genfun\paren{\{b\}\cup \set U(\set Q')}$, and so by Lemma \ref{recombining elements preserves conjugal property for third parties} we also know that every pair in $\{\lst P'_i\}\backslash\{(a,b)\}$ is also conjugal to $\set T''$.

Next let $\lst R$ be equal to $\lst P'$ with its first $k$ elements dropped, $\lst f^{(1)}$ be a sequence of $1$'s repeated $|\lst R|$ times, and
$$\lst f^{(2)}_i :=
\begin{cases}
1 & p_2(\lst R)\in\set X', \{p_2(\lst R_i),x\} = 0; \\
0 & \text{otherwise}.
\end{cases}
$$
Let $\lst R'$ and $b''$ be the objects named in Lemma \ref{multiple-recombinations} given the pair $(a,b):=(a,b')$, the sequences $\lst f^{(l)}$ and the sequence $\lst R$, and let
$$\lst P :=
\begin{cases}
\lst P'_i & i < k \\
(a,b'') & i=k \\
\lst R_{i-k}' & \text{otherwise}.
\end{cases}
$$
By Lemma \ref{multiple-recombinations}, we know that $(a,b'')$ and every pair in $\lst R'$ are conjugal to
$$\left[\set T''\backslash\paren{\{a,b'\} \cup \set U(\set R)}\right]\cup \{a,b''\} \cup \set U(\set Q')=\set T,$$
and we also know that $\genfun\paren{\{b''\}\cup \set U(\set R')}=\genfun\paren{\{b'\}\cup \set U(\set R)}$;
by Lemma \ref{recombining elements preserves conjugal property for third parties} we thus conclude that every pair in $\lst P$ and $\set Q$ is also conjugal to $\set T$.  We have thus proven that conclusion \ref{pairs are all conjugal to T} holds, and we furthermore note that by combining our results we see that $\genfun\paren{\set U(\lst P)\cup \set U(\set Q)}=\genfun\paren{\set U(\lst P')\cup \set U(\set Q')}=(\genfun\circ\set U)(\set L)$, and so conclusion \ref{P and Q generate the same set as L} also holds.

Observe that $b''$ is a product of elements from $\{p_1(\lst P_i') : i \ge k\}_i\cup\set U(\set Q)$, and by the inductive hypothesis and specifically properties \ref{first members are all in O} and \ref{operators in X are no worse than the best qubit in P} we see that it is a product of elements $\set Y\subseteq\set O'$ such that $\min_{y\in\set Y}\om_{\set S}(y)=\om_{\set S}(b)$.  Hence, by Lemma \ref{replacing element with product preserves unimprovability} we conclude that $\om_{\set S}(b)=\om_{\set S}(b'')$, the set $\set O'':=\paren{\set O'\backslash\{b\}}\cup\{b''\}$ is an unimprovable set and $\genfun(\set O')=\genfun(\set O'')$;  from the last equality we see from Lemma \ref{recombining elements preserves extension} that $\set O''$ extends to $\set X'$.

Note that, by construction, every element in $\set X$ is either an element of $\set X'$ or an element of $\set X'$ multiplied by $x\in\set X'$.  Thus $\set X\subseteq \genfun(\set X')$, and so by Lemma \ref{recombining extension elements preserves extension} we see that $\set O''$ extends to $\set X$.  Furthermore, by Lemma \ref{combinations-can't-make-things-worse} and conclusion \ref{operators in X are no worse than the best qubit in P} from the inductive hypothesis we see that $\min_{y\in\set X}(y) \ge \min_{y\in\set X'}(y) \ge m_{\set S}(\lst P'_{|\lst P'|})$.

Note that there are only two kinds of changes from $\lst P'$ to $\lst P$:  the change in the first member of a pair at $k$, and changes in the second members of pairs from being an element of $\set X'$ to an element of $\set X$.  Since we have already shown that $(\om_{\set S}\circ p_1)(\lst P_k) = (\om_{\set S}\circ p_1)(\lst P_k')$ and $\min_{y\in\set X}(y) \ge m_{\set S}(\lst P'_{|\lst P'|})$, we conclude that Thus, $m(\lst P_i)=m(\lst P_i')$ for every $1 \le i \le |\lst P'$, and so $M^{(|\lst P|)}(\lst P)=M^{(|\lst P'|)}(\lst P')$, which means in particular that the sequence $M^{(|\lst P|)}(\lst P)=M^{(|\lst P'|)}(\lst P')$ is ordered and hence $\lst P$ is a choice of qubits.  Furthermore, since in particular $m_{\set S}(\lst P'_{|\lst P'|})=m_{\set S}(\lst P_{|\lst P|})$, we conclude that $\min_{y\in\set X}(y)\gem_{\set S}(\lst P_{|\lst P|})$, proving property \ref{operators in X are no worse than the best qubit in P}.

Let $\set O:=\set O''\cup\{x\}$.  Since $|\set O''|=|\set O|=n-1$, we see that $|\set O|=|\set O''|+1=n$, and so conclusion \ref{size of O matches arbitrarily chosen integer} holds.  Since we replaced $b$ with $b''$ in going from $\set O'$ to $\set O$, and $b$ is the only first member of any pair that changed from $\lst P'$ to $\lst P$, we conclude that conclusion \ref{first members are all in O} from the inductive hypothesis continues to hold for $\set O$ and $\lst P$.

It remains only to show that $\set O$ extends to $\set X$;  recall that we have already shown that $\set O''$ extends to $\set X$.

\end{description}

\end{proof}
%@-node:gcross.20100127211423.1361:Theorem: Creation of certificate
%@-others
%@-node:gcross.20090511123440.3:Optimal generators
%@-node:gcross.20090423002455.2:Algorithm
%@-others

\end{document}
%@nonl
%@-node:gcross.20090405101642.3:@thin CodeQuest.tex
%@-leo
